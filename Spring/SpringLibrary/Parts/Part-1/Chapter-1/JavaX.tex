\section{JavaX 注解}

这里放一些不是 Spring 系但有关联,在扩展包(Java Extension)中出现的常用注解。

\subsection{Validation 注解}

\subsubsection{@NotNull,@NotEmpty,@NotBlank}

这三个注解均用来哦按段是否为空,分别为:
\begin{itemize}
    \item @NotNull: 属性不可以为 null,但可以为空串。
    \item @NotEmpty: 属性不可以为 null, 且不可以为空串(长度大于0)。
    \item @NotBlank: 只能作用于 String 类型的属性上,属性不可以为 null,且 trim() 后不可以为空串。
\end{itemize}

其中 @NotNull 源代码如下:
\begin{Java}
@Target({ METHOD, FIELD, ANNOTATION_TYPE, CONSTRUCTOR, PARAMETER, TYPE_USE })
@Retention(RUNTIME)
@Repeatable(List.class)
@Documented
@Constraint(validatedBy = { })
public @interface NotNull {
    String message() default "{javax.validation.constraints.NotNull.message}";
    Class<?>[] groups() default { };
    Class<? extends Payload>[] payload() default { };
    @Target({ METHOD, FIELD, ANNOTATION_TYPE, CONSTRUCTOR, PARAMETER, TYPE_USE })
    @Retention(RUNTIME)
    @Documented
    @interface List {
        NotNull[] value();
    }
}
\end{Java}

\subsubsection{@Size,@Length,@Max,@Min}

这三个注解均用于判断数值大小/字符串长度:
\begin{itemize}
    \item @Min: 验证 Number 和 String 对象是否大于等于指定的值。
    \item @Max: 验证 Number 和 String 对象是否小于等于指定的值。
    \item @Size(min=,max=): 验证对象(Array,Collection,Map,String)长度是否在给定范围内。
    \item @Length(min=,max=): 验证字符串长度是否在给定范围内。
\end{itemize}

其中 @Size 源代码如下:
\begin{Java}
@Target({ METHOD, FIELD, ANNOTATION_TYPE, CONSTRUCTOR, PARAMETER, TYPE_USE })
@Retention(RUNTIME)
@Repeatable(List.class)
@Documented
@Constraint(validatedBy = { })
public @interface Size {
    String message() default "{javax.validation.constraints.Size.message}";
    Class<?>[] groups() default { };
    Class<? extends Payload>[] payload() default { };
    /**
     * @return size the element must be higher or equal to
     */
    int min() default 0;
    /**
     * @return size the element must be lower or equal to
     */
    int max() default Integer.MAX_VALUE;
    @Target({ METHOD, FIELD, ANNOTATION_TYPE, CONSTRUCTOR, PARAMETER, TYPE_USE })
    @Retention(RUNTIME)
    @Documented
    @interface List {
        Size[] value();
    }
}
\end{Java}

\subsubsection{@Digits}

被注解的元素必须为一个数字,其值必须在可接受的范围内。主要参数有两个:
\begin{itemize}
    \item integer: 整数部分位数。
    \item fraction: 小数部分位数。
\end{itemize}

\begin{Java}
@Digits(integer = 3,fraction = 0)
private String ccCVV;
\end{Java}

\subsubsection{@Valid}

@Valid注解用于校验。\footnote{参考: \url{https://blog.csdn.net/gaojp008/article/details/80583301}}

\begin{Java}
@Target({ METHOD, FIELD, CONSTRUCTOR, PARAMETER, TYPE_USE })
@Retention(RUNTIME)
@Documented
public @interface Valid {
}
\end{Java}

使用校验,首先要在实体类相应字段上添加用于校验的注解。如: @Min。其次在 Controller 层的方法的要校验的参数上添加 @Valid 注解,并且需要传入 BindingResult 对象,用于获取校验失败情况下的反馈信息,代码如下:

\begin{Java}
@PostMapping("/girls")  
public Girl addGirl(@Valid Girl girl, BindingResult bindingResult) {  
    if(bindingResult.hasErrors()){  
        System.out.println(bindingResult.getFieldError().getDefaultMessage());  
        return null;  
    }  
    return girlResposity.save(girl);  
}
\end{Java}

\newpage