\section{SpringBoot 注解}
\subsection{SpringBoot 常见注解}
\subsubsection{@SpringBootApplication}

@SpringBootApplication 表明这是一个 Spring 引导应用程序,默认 SpringBoot 项目的启动类会被该注解修饰。

\begin{Java}
@Target(ElementType.TYPE)
@Retention(RetentionPolicy.RUNTIME)
@Documented
@Inherited
@SpringBootConfiguration
@EnableAutoConfiguration
@ComponentScan(excludeFilters = { 
    @Filter(type = FilterType.CUSTOM,classes = TypeExcludeFilter.class), 
    @Filter(type = FilterType.CUSTOM, classes =AutoConfigurationExcludeFilter.class) })
public @interface SpringBootApplication {
    // ... 此处省略源码
}
\end{Java}

可以看出,该注解是由三个注解组成,这里做简单说明,细节请查询相应的注解:
\begin{itemize}
    \item @ComponentScan: 自动扫描并加载符合条件的组件。
    \item @EnableAutoConfiguration: 借助@Import的支持,收集和注册特定场景相关的bean定义。
    \item @SpringBootConfiguration: 标注当前类是配置类,并会将当前类内声明的一个或多个以@Bean注解标记的方法的实例纳入到spring容器中,并且实例名就是方法名。
\end{itemize}

参考文献:
\begin{itemize}
    \item CSDN: \url{https://blog.csdn.net/qq_28289405/article/details/81302498}
\end{itemize}

\newpage