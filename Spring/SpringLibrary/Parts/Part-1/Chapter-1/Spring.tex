\section{Spring Framework 注解}
\subsection{Spring Bean 基础注解}

这节的前四个个注解功能基本都是一样的,都是将类作为 bean 注入到 spring 容器中进行管理,只不过它们使用的场景不同。后几个注解是对前几个注解的修饰。

选择上,如果确定是 MVC 的哪一层,就选择对应的具体注解(@Controller, @Service, @Repository)标识,如果不确定,但又知道它是各组件,就用 @Component 标识。这四个注解除了可以明确层次关系,没有其他区别。

\subsubsection{@Component}

@Component 标注一个普通的组件类,通知 Spring 将类纳入到 Spring Bean 容器中并进行管理。默认定义的 bean 是单例的。可以通过 @Component(“beanName”) @Scope(“prototype”) 改变,下面同样。

\begin{Java}
@Target(ElementType.TYPE)
@Retention(RetentionPolicy.RUNTIME)
@Documented
@Indexed
public @interface Component {
    String value() default "";
}
\end{Java}

\subsubsection{@Controller}

对应 Spring MVC 控制层,主要用户接受用户 http 请求并调用 Service 层返回数据给前端页面。

\begin{Java}
@Target(ElementType.TYPE)
@Retention(RetentionPolicy.RUNTIME)
@Documented
@Component
public @interface Controller {
    @AliasFor(annotation = Component.class)
    String value() default "";
}
\end{Java}

\subsubsection{@Service}

对应 Spring MVC 业务层。主要用于获取 pojo 层的的数据并进行业务处理,

\begin{Java}
@Target(ElementType.TYPE)
@Retention(RetentionPolicy.RUNTIME)
@Documented
@Component
public @interface Service {
    @AliasFor(annotation = Component.class)
    String value() default "";
}
\end{Java}

\subsubsection{@Repository}

@Repository 对应持久层(pojo)。主要用于直接和数据库交互。

\begin{Java}
@Target(ElementType.TYPE)
@Retention(RetentionPolicy.RUNTIME)
@Documented
@Component
public @interface Repository {
    @AliasFor(annotation = Component.class)
    String value() default "";
}
\end{Java}

\subsubsection{@Scope}

@Scope 注解用于指定 Bean 的作用范围,也即采用的设计模式。

\begin{Java}
@Target({ElementType.TYPE, ElementType.METHOD})
@Retention(RetentionPolicy.RUNTIME)
@Documented
public @interface Scope {
    @AliasFor("scopeName")
	String value() default "";
	@AliasFor("value")
	String scopeName() default "";
    ScopedProxyMode proxyMode() default ScopedProxyMode.DEFAULT;
}
\end{Java}
一般的有两个选项,"singleton" 与 "prototype",分别代表单例设计模式与原型设计模式。默认采用单例模式。

\subsubsection{@PostConstrucor 与 @PreDestroy}

顾名思义,这两个注解作用的生命周期分别是: 构造方法后和销毁前。对应了 Bean 生命周期的 init-method 和 destroy-method。

\begin{Java}
@Documented
@Retention (RUNTIME)
@Target(METHOD)
public @interface PostConstruct

@Documented
@Retention (RUNTIME)
@Target(METHOD)
public @interface PreDestroy

\end{Java}

\subsubsection{@Bean}

@Bean 主要用于将一个方法地返返回值注册为 Bean:

\begin{Java}
@Target({ElementType.METHOD, ElementType.ANNOTATION_TYPE})
@Retention(RetentionPolicy.RUNTIME)
@Documented
public @interface Bean
\end{Java}

为什么要这样设计呢,主要是为了管理外部的 Bean:

\begin{Java}
@Bean
public DataSource getDataSource() {
    DruidDataSource di = new DruidDataSource();
    ds.setDriverClassName("com.mysql.jdbc.Driver");
    ...
    return ds;
}
\end{Java}

同时,@Bean 也是自动装配的,如果有形参,容器会自动注入对应类型的形参。

\subsection{Spring Bean 配置注解}

\subsubsection{@Configuration}

声明当前类为配置类,相当于 xml 形式的 Spring 配置。

\begin{Java}
@Target(ElementType.TYPE)
@Retention(RetentionPolicy.RUNTIME)
@Documented
@Component
public @interface Configuration {
    @AliasFor(annotation = Component.class)
    String value() default "";
    boolean proxyBeanMethods() default true;
}
\end{Java}

\subsubsection{@ComponentScan}

顾名思义,用来扫描 Component 并批量注册 Bean,默认情况下扫描当前包及子包的 Component,可以自定义扫描位置\footnote{延申文献: \url{https://zhuanlan.zhihu.com/p/520827986}}:

\begin{Java}
@Retention(RetentionPolicy.RUNTIME)
@Target(ElementType.TYPE)
@Documented
@Repeatable(ComponentScans.class)
public @interface ComponentScan {
    @AliasFor("basePackages")
    String[] value() default {};
    ......
}
\end{Java}

\subsubsection{@PropertySource}

@PropertySource 注解用于绑定 properties 文件:

\begin{Java}
@Target(ElementType.TYPE)
@Retention(RetentionPolicy.RUNTIME)
@Documented
@Repeatable(PropertySources.class)
public @interface PropertySource {
    String[] value();
    ......
}
\end{Java}

一般的 @PropertySource 作用在 Config 文件上\footnote{延申文献: \url{https://blog.csdn.net/tenghu8888/article/details/119791417}}。

\subsubsection{@Import}

@import 注解用于导入别的注解,当有多个 Config 时,一般在 Spring 的 Config 中导入其他配置文件。

\begin{Java}
@Target(ElementType.TYPE)
@Retention(RetentionPolicy.RUNTIME)
@Documented
public @interface Import {
    Class<?>[] value();
}
\end{Java}

举个例子:

\begin{Java}
@Configuration
@Import
public class SpringConfig
\end{Java}

\subsection{Spring DI 注解}

前面 Spring 已经完成了 Bean 的标识与扫描,但这仅仅是将 Bean 装入了 IoC 容器,下面注解实现了 DI。

\subsubsection{@Autowired}

@Autowired 注解顾名思义,进行自动装配,是一个功能十分强大,实现比较复杂,有争议的注解。

\begin{Java}
@Target({ElementType.CONSTRUCTOR, ElementType.METHOD, ElementType.PARAMETER, ElementType.FIELD, ElementType.ANNOTATION_TYPE})
@Retention(RetentionPolicy.RUNTIME)
@Documented
public @interface Autowired {
    boolean required() default true;
}    
\end{Java}

默认情况下,@Autowired 根据类型实现自动装配,构造方法注入和 setter 注入都可以实现。@Autowired 可以直接作用在属性上,而不直接使用 setter 方法注入,它将使用暴力反射的方式将对应的类型注入到属性中\footnote{延申文献: \url{https://blog.csdn.net/Weixiaohuai/article/details/123005140}}。注意,既然使用了暴力反射,就要提供 bean 的无参构造方法。

@Autowired 默认使用也推荐使用类型注入,但类型注入必然会带来问题: 同类型二义性,不知道用哪个。

\subsubsection{@Qualifier}

@Qualifier 用于按名称注入,他必须依赖于 @Autowired 注解。

\begin{Java}
@Target({ElementType.FIELD, ElementType.METHOD, ElementType.PARAMETER, ElementType.TYPE, ElementType.ANNOTATION_TYPE})
@Retention(RetentionPolicy.RUNTIME)
@Inherited
@Documented
public @interface Qualifier {
    String value() default "";
}
\end{Java}

使用时,必须先给出 @Autowired 注解:

\begin{Java}
@Autowired
@Qualifier("bookDao")   // 对应某个 bean 名为 bookDao
private BookDao bookDao;
\end{Java}

当然这样有个问题,耦合度高!!!

\subsubsection{@Value}

注意这里是 Spring Framework 里的 @Value 不是 lombok 的 @Value。

@Autowired 有一个缺陷,他只能注入引用类型,注入基本类型\footnote{想念 Python}要依靠 @Value:

\begin{Java}
@Target({ElementType.FIELD, ElementType.METHOD, ElementType.PARAMETER, ElementType.ANNOTATION_TYPE})
@Retention(RetentionPolicy.RUNTIME)
@Documented
public @interface Value {
    String value();
}
\end{Java}

@Value 一般用于注入 properties 文件的内容。需要结合 @PropertySource 注解使用。

\begin{Java}
@PropertySource("classpath:jdbc.properties")    // classpath: 可以不加
public class JdbcProperties {
    ......
}

@Repository
public class BookDao {
    @Value(${name})     // jdbc.properties 存在字段和 ${} 内容相同
    private String name;
}
\end{Java}

\subsection{Spring AOP 注解}

这小节的注解需要下面依赖:

\begin{xml}
<dependency>
    <groupId>org.aspectj</groupId>
    <artifactId>aspectjweaver</artifactId>
</dependency>
\end{xml}

\subsubsection{@EnableAspectJAutoProxy}

该注解用于启动对应配置类下的 AOP,具体表现为识别 @Aspect 注解\footnote{拓展: \url{https://blog.csdn.net/yuan882696yan/article/details/115359291}}。

\begin{Java}
@Target(ElementType.TYPE)
@Retention(RetentionPolicy.RUNTIME)
@Documented
@Import(AspectJAutoProxyRegistrar.class)
public @interface EnableAspectJAutoProxy {
    boolean proxyTargetClass() default false;
    boolean exposeProxy() default false;
}    
\end{Java}

\subsubsection{@Aspect}

把当前类标识为一个切面供容器读取。注意在只用它之前要确保类已经被 @Component 标注为 Bean。

\begin{Java}
@Retention(RetentionPolicy.RUNTIME)
@Target(ElementType.TYPE)
public @interface Aspect {
    String value() default "";
}
\end{Java}

\subsubsection{@Pointcut}

该注解用于指定切入点。

\begin{Java}
@Retention(RetentionPolicy.RUNTIME)
@Target(ElementType.METHOD)
public @interface Pointcut {
    String value() default "";
    String argNames() default "";
}
\end{Java}

切入点定义依托于一个不具有实际意义的方法进行,及无参数,无返回值,方法体无实际逻辑。可以理解为,运行到该方法时,需要加功能(Advice)。

\textbf{切入点表达式}\footnote{这里只讲最常用的语法。}: 要进行增强的方法的描述方式。 一个切入点表达式的标准格式如下:

\begin{center}
动作关键字(访问修饰符  返回值 包名.类/接口名.方法名(参数) 异常名)
\end{center}

例如: \texttt{execution(public User com.pionpill.UserService.findById(int))}

一般的,动作关键字都是 \texttt{execution},访问修饰符都是 \texttt{public} 可以省略,大多数情况下没有异常,可以省略。

切入点表达式中可以使用通配符简化书写,但是这样会降低性能和可读性,应该减少或者不使用,这里仅作说明:

\begin{itemize}
    \item \textbf{*}: 表示任意类型返回值/包名/类名/方法名,但至少一个。
    \item \textbf{..}: 表示任意多个参数,任意包地址,可以没有。
    \item \textbf{+}: 代表子类类型。
\end{itemize}

同时的,在切入点表达式中可以使用``与 或 非''三种运算符来组合切入点表达式。

\begin{Java}
@Pointcut("@execution(org.com.common.aspect.annotation.PermissionData)")
public void pointCut() { }
\end{Java}

\subsubsection{Advice}

Advice 也即要进行增强处理的方法,它包括五个注解。这五个注解的使用方式类似。都需要加入切入点方法作为参数。假设切入点方法如下:

\begin{Java}
@Pointcut("execution (void ...)")
public void pt() { }
\end{Java}

\paragraph*{@Before, @After} 在切入点之前/后执行

\begin{Java}
@Retention(RetentionPolicy.RUNTIME)
@Target(ElementType.METHOD)
public @interface Before {
    String value();
    String argNames() default "";
}
\end{Java}

这两个注解使用起来非常简单:

\begin{Java}
@Before("pt()")
public void before () {
    System.out.println("Before execution ...");
}
\end{Java}

\paragraph*{@Around} 这是最重要也最常用的 Advice 注解。它对应的方法需要一个 ProceedingJoinPoint 类型的参数用于标记切入点方法执行的位置。

\begin{Java}
@Around("pt()")
public void around(ProceedingJoinPoint pjp) throws Throwable{
    System.out.println("Before execution ...");
    pjp.proceed();  // 对原始方法的调用
    System.out.println("After execution ...");
}
\end{Java}

这里需要抛出异常,因为不确定原始方法究竟有没有异常,所以强制抛出异常。

这里有个注意点,如果在 around 中返回了其他值,将覆盖原始方法的返回值。

\paragraph*{@AfterReturning, @AfterThrowing} 分别用于在返回值之后和抛出异常后执行,不是很常用。

\begin{Java}
@Retention(RetentionPolicy.RUNTIME)
@Target(ElementType.METHOD)
public @interface AfterReturning {
    String value() default "";
    String pointcut() default "";
    String returning() default "";
    String argNames() default "";
}

@Retention(RetentionPolicy.RUNTIME)
@Target(ElementType.METHOD)
public @interface AfterThrowing {
    String value() default "";
    String pointcut() default "";
    String throwing() default "";
    String argNames() default "";
}
\end{Java}

\subsection{Spring 事务注解}

这小节的注解需要下面依赖:

\begin{xml}
<dependency>
    <groupId>org.springframework</groupId>
    <artifactId>spring-tx</artifactId>
</dependency>
\end{xml}

\subsubsection{@EnableTransactionManagement}

用于开启注解式事务驱动。在配置类上开启了这个,才能使用其他事务注解。

\begin{Java}
@Target(ElementType.TYPE)
@Retention(RetentionPolicy.RUNTIME)
@Documented
@Import(TransactionManagementConfigurationSelector.class)
public @interface EnableTransactionManagement {
    boolean proxyTargetClass() default false;
    AdviceMode mode() default AdviceMode.PROXY;
    int order() default Ordered.LOWEST_PRECEDENCE;
}
\end{Java}

\subsubsection{@Transactional}

该注解用于添加 Spring 事务管理。一般不用于开在实现类中,而是接口中。

\begin{Java}
@Target({ElementType.TYPE, ElementType.METHOD})
@Retention(RetentionPolicy.RUNTIME)
@Inherited
@Documented
public @interface Transactional {
    @AliasFor("transactionManager")
    String value() default "";
    @AliasFor("value")
    String transactionManager() default "";
    String[] label() default {};
    Propagation propagation() default Propagation.REQUIRED;
    Isolation isolation() default Isolation.DEFAULT;
    int timeout() default TransactionDefinition.TIMEOUT_DEFAULT;
    String timeoutString() default "";
    boolean readOnly() default false;
    Class<? extends Throwable>[] rollbackFor() default {};
    String[] rollbackForClassName() default {};
    Class<? extends Throwable>[] noRollbackFor() default {};
    String[] noRollbackForClassName() default {};
}
\end{Java}

事务相关的配置如下:

\begin{table}[H]
    \centering
    \caption{事务相关配置}
    \label{table:事务相关配置}
    \setlength{\tabcolsep}{4mm}
    \begin{tabular}{l|lll}
        \toprule
        \textbf{属性} & \textbf{作用} & \textbf{示例} \\
        \midrule
        readOnly & 设置是否为只读事务 & true: 只读 \\
        timeout & 设置事务超时事件 & timeout = -1: 永不超时 \\
        rollbackFor & 设置事务回滚异常 & rollbackFor = (NullPointException.class) \\
        noRollbackFor & 设置事务不回滚异常 & noRollbackFor = (NullPointException.class) \\
        propagation & 设置事务传播行为 & ... \\
        \bottomrule
    \end{tabular}
\end{table}

默认情况下,Spring 事务只对 Unchecked Exception (Error, Runtime Exception)异常进行回滚。如果需要添加其他异常或者将某些异常剔除,需要用到 rollBackFor 和 noRollBackFor 属性。propagation 用于传播事务行为,我们将主事务(需要执行的那个事务)称为事务管理员,被加入的其他事务成为事务协调员。

\begin{table}[H]
    \centering
    \small
    \caption{事务传播行为}
    \label{table:事务传播行为}
    \setlength{\tabcolsep}{4mm}
    \begin{tabular}{l|c|c}
        \toprule
        \textbf{传播属性} & \textbf{事务管理员} & \textbf{事务协调员} \\
        \midrule
        \multirow{2}{*}{REQUIRED(默认)} & 开启 & 加入 \\
         & 无 & 新建 \\
        \midrule
        \multirow{2}{*}{REQUIRED\_NEW} & 开启 & 新建 \\
        & 无 & 新建 \\
        \midrule
        \multirow{2}{*}{SUPPORTS} & 开启 & 加入 \\
        & 无 & 无 \\
        \midrule
        \multirow{2}{*}{NOT\_SUPPOETED} & 开启 & 无 \\
        & 无 & 无 \\
        \midrule
        \multirow{2}{*}{MANDATORY} & 开启 & 加入 \\
        & 无 & ERROR \\
        \midrule
        \multirow{2}{*}{NEVER} & 开启 & ERROR \\
        & 无 & 无 \\
        \midrule
        NESTED & \multicolumn{2}{l}{设置 savePoint, 事务将回滚到此处。} \\
        \bottomrule
    \end{tabular}
\end{table}

\newpage