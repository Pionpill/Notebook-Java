\section{Spring MVC 注解}

\subsection{MVC 控制层相关注解}

\subsubsection{@RequestMapping}

作用: 将 Web 请求与请求处理类中的方法进行映射

\begin{Java}
@Target({ElementType.TYPE, ElementType.METHOD})
@Retention(RetentionPolicy.RUNTIME)
@Documented
@Mapping
public @interface RequestMapping {
    // 省略具体实现
}
\end{Java}

它有以下的六个配置属性:
\begin{itemize}
    \item value: 映射请求的 URL 或者别名。
    \item method: 兼容HTTP的方法名
    \item params: 根据HTTP参数的存在、缺省或值对请求进行过滤
    \item header: 根据HTTP Header的存在、缺省或值对请求进行过滤
    \item consume: 设定在HTTP请求正文中允许使用的媒体类型
    \item product: 在HTTP响应体中允许使用的媒体类型
\end{itemize}

在使用@RequestMapping之前,请求处理类还需要使用 @Controller 进行标记。

\begin{Java}
@Controller
@RequestMapping("/home")
public class HomeController {
    ......
}
\end{Java}

\paragraph*{@GetMapping}

用于于处理HTTP GET请求,并将请求映射到具体的处理方法中。具体来说,@GetMapping是一个组合注解,它相当于是@RequestMapping(method=RequestMethod.GET)的快捷方式。下面几个映射注解也类似。

\begin{Java}
@Target(ElementType.METHOD)
@Retention(RetentionPolicy.RUNTIME)
@Documented
@RequestMapping(method = RequestMethod.GET)
public @interface GetMapping {
    // 省略具体实现
}
\end{Java}

给个例子,下面几个不给了:

\begin{Java}
@Controller
@RequestMapping("/home")
public class HomeController {
    @GetMapping("/users")
    public List<User> findAllUsers() {
        List<User> users = userService.findAll();
        return users;
    }
    ......
}
\end{Java}

\paragraph*{@PostMapping}

注解用于处理HTTP POST请求。

\begin{Java}
@Target(ElementType.METHOD)
@Retention(RetentionPolicy.RUNTIME)
@Documented
@RequestMapping(method = RequestMethod.POST)
public @interface PostMapping {
    // 省略具体实现
}
\end{Java}

\paragraph*{@PutMapping}

注解用于处理HTTP PUT请求,

\begin{Java}
@Target(ElementType.METHOD)
@Retention(RetentionPolicy.RUNTIME)
@Documented
@RequestMapping(method = RequestMethod.PUT)
public @interface PutMapping {
    // 省略具体实现
}
\end{Java}

\paragraph*{@DeleteMapping}

注解用于处理HTTP Delete请求,

\begin{Java}
@Target(ElementType.METHOD)
@Retention(RetentionPolicy.RUNTIME)
@Documented
@RequestMapping(method = RequestMethod.DELETE)
public @interface DeleteMapping {
    // 省略具体实现
}
\end{Java}

\paragraph*{@PatchMapping}

注解用于处理HTTP PATCH请求,

\begin{Java}
@Target(ElementType.METHOD)
@Retention(RetentionPolicy.RUNTIME)
@Documented
@RequestMapping(method = RequestMethod.DELETE)
public @interface DeleteMapping {
    // 省略具体实现
}
\end{Java}

\subsubsection{@ResponseBody}

用于将Controller的方法返回的对象,通过 springmvc 提供的HttpMessageConverter接口转换为指定格式的数据如:json、xml等,通过Response响应给客户端。

\begin{Java}
@Target({ElementType.TYPE, ElementType.METHOD})
@Retention(RetentionPolicy.RUNTIME)
@Documented
public @interface ResponseBody { }    
\end{Java}

\subsubsection{@RequestParam}

该注解用于将请求参数绑定到控制器的方法参数上:

\begin{Java}
@Target(ElementType.PARAMETER)
@Retention(RetentionPolicy.RUNTIME)
@Documented
public @interface RequestParam {
    @AliasFor("name")
    String value() default "";
    @AliasFor("value")
    String name() default "";
    boolean required() default true;
    String defaultValue() default ValueConstants.DEFAULT_NONE;
}
\end{Java}

在参数上加入该注解后有如下效果:
\begin{itemize}
    \item 不加该注解的前端和后端参数名需要一致。
    \item 加该注解的参数为必传参数,不加为非必传,也可通过 \texttt{required = false} 设置为非必传。
    \item 可以通过 \texttt{value} 指定参数名。
    \item 可以通过 \texttt{defaultValue} 指定参数默认值。
\end{itemize}

\subsubsection{@RequestBody}

用于将请求中请求体包含的数据传递给请求参数,一个处理器方法只能使用一次:

\begin{Java}
@Target(ElementType.PARAMETER)
@Retention(RetentionPolicy.RUNTIME)
@Documented
public @interface RequestBody {
    boolean required() default true;
}
\end{Java}


\subsubsection{@EnableWebMvc}

该注解可以开启 SpringMVC 多项辅助功能:

\begin{Java}
@Retention(RetentionPolicy.RUNTIME)
@Target(ElementType.TYPE)
@Documented
@Import(DelegatingWebMvcConfiguration.class)
public @interface EnableWebMvc { }
\end{Java}

\subsubsection{@DateTimeFormat}

该注解用于将传入的 String 类型参数转换为 Data 类型,用于设定日期时间型数据格式。

\begin{Java}
@Documented
@Retention(RetentionPolicy.RUNTIME)
@Target({ElementType.METHOD, ElementType.FIELD, ElementType.PARAMETER, ElementType.ANNOTATION_TYPE})
public @interface DateTimeFormat {
    String style() default "SS";
    ISO iso() default ISO.NONE;
    String pattern() default "";
    String[] fallbackPatterns() default {};
    enum ISO { DATE, TIME, DATE_TIME, NONE }
}    
\end{Java}

举个例子:

\begin{Java}
@DateTimeFormat(pattern="yyyy-MM-dd HH:mm:ss")
private Date date;
\end{Java}

\subsection{MVC 视图相关注解}

\subsubsection{@ModelAttribute}

@ModelAttribute主要的作用是将数据添加到模型对象中,用于视图页面显示。@ModelAttribute注释的位置不同,和其他注解一起使用时有很多种用法。\footnote{参考文献: \url{https://blog.csdn.net/yue_xx/article/details/105740360}}

\begin{Java}
@Target({ElementType.PARAMETER, ElementType.METHOD})
@Retention(RetentionPolicy.RUNTIME)
@Documented
public @interface ModelAttribute {
    @AliasFor("name")
    String value() default "";
    @AliasFor("value")
    String name() default "";
    boolean binding() default true;
}
\end{Java}

主要用途可分为注解在方法上和注解在参数上:
\begin{itemize}
    \item 用在方法上,@ModelAttribute 注解的方法会在 Controller 每个方法被执行前调用。根据返回值不同也有区分。
    \begin{itemize}
        \item void 方法: 一般会在方法的参数中使用 Model 参数,在方法体内将模型数据添加到模型对象中。
\begin{Java}
@ModelAttribute 
public void NoneReturn(@RequestParam String data, Model model) { 
    model.addAttribute("指定一个名称",data); 
}
\end{Java}
        \item 具体类型方法: 一般用@ModelAttribute的value属性指定model属性的名称。model属性对应的对象就是方法的返回值。不指定名称,则model属性名就会默认是返回类型的首字母小写。
\begin{Java}
@ModelAttribute(name = "tacoOrder")
public TacoOrder order() {
    return new TacoOrder();
}
\end{Java}
        \item @ModdelAttribute和@RequestMapping共同注解一个方法时: 此时方法的返回值并不是表示一个视图名称,而是model属性的值,此时的视图名称就是@RequestMapping中指定的访问路径的最后一层去掉扩展名。
\begin{Java}
@RequestMapping(value = "指定一个访问路径") 
@ModelAttribute("指定一个名称") 
public String Fix() { 
    return "猛虎蔷薇"; 
} 
\end{Java}
    \end{itemize}
    \item 标记在方法的参数上,会将客户端传递过来的参数按名称注入到指定对象中,并且会将这个对象自动加入ModelMap中,便于View层使用。
\end{itemize}

\subsubsection{@SessionAttributes}

若希望在多个请求之间共用数据,则可以在控制器类上标注一个 @SessionAttributes,配置需要在session中存放的数据范围,Spring MVC将存放在model中对应的数据暂存到 HttpSession 中。

\begin{Java}
@Target({ElementType.TYPE})
@Retention(RetentionPolicy.RUNTIME)
@Inherited
@Documented
public @interface SessionAttributes {
    @AliasFor("names")
    String[] value() default {};
    @AliasFor("value")
    String[] names() default {};
    Class<?>[] types() default {};
}
\end{Java}








\newpage