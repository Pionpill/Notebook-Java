\documentclass{PionpillNote-book}
\usetikzlibrary {intersections,through,arrows.meta,graphs,shapes.misc,positioning,shapes.misc,positioning,calc}
\usetikzlibrary{animations}
\usetikzlibrary {shapes.geometric}
\usetikzlibrary {animations}
\usetikzlibrary {shapes.multipart}
\usetikzlibrary {positioning}
\usetikzlibrary {fit,shapes.geometric}
\usetikzlibrary {automata}
\usetikzlibrary {quotes}
\usetikzlibrary {matrix}
\usetikzlibrary {backgrounds}
\usetikzlibrary {scopes}
\usetikzlibrary {calc}
\usetikzlibrary {intersections}
\usetikzlibrary {svg.path}
\usetikzlibrary {decorations}
\usetikzlibrary {patterns}
\usetikzlibrary {decorations.pathmorphing}
\usetikzlibrary {shadows}
\usetikzlibrary {bending}

\title{Spring Principle}
\author{
    Pionpill \footnote{笔名:北岸,电子邮件:673486387@qq.com,Github:\url{https://github.com/Pionpill}} \\
    本文档为作者学习 Spring 理论时的笔记。\\
}

\date{\today}

\begin{document}

\pagestyle{plain}
\maketitle

\noindent\textbf{前言:}

笔者为软件工程系在校本科生,有计算机学科理论基础(操作系统,数据结构,计算机网络,编译原理等),本人在撰写此笔记时已有 Java 开发经验,基础知识不再赘述。

本书第一节主要参考了 《Spring 揭秘》\footnote{《Spring 揭秘》: 王福强,2009 版}。很遗憾这本书没有再版,市面上也找不到能够替代该书的书籍。其中 AOP 部分参考了CSDN 博主滕青山的笔记\footnote{笔记地址: \url{https://blog.csdn.net/qq_34626094/category_11731455.html}}。

本文重点是 Spring 原理,入门及实战可以参考网上的一些视频教程。

本人的编写及开发环境如下:
\begin{itemize}
    \item Java: Java11
    \item SpringBoot: 2.7.3
    \item OS: Windows11 
    \item MySQL: 8.0.3
\end{itemize}

\date{\today}
\newpage

\tableofcontents

\newpage

\setcounter{page}{1} 
\pagestyle{fancy}

\part{Spring FrameWork}
\chapter{Spring 的 IoC 容器}
\import{Parts/Part-FrameWork/Chapter-IoC}{IoC-Basic.tex}
\import{Parts/Part-FrameWork/Chapter-IoC}{IoC-ServiceProvider.tex}
\import{Parts/Part-FrameWork/Chapter-IoC}{BeanFactory.tex}
\import{Parts/Part-FrameWork/Chapter-IoC}{ApplicationContext.tex}
\chapter{Spring 的 AOP 框架}
\import{Parts/Part-FrameWork/Chapter-AOP}{AOP-Basic.tex}
\import{Parts/Part-FrameWork/Chapter-AOP}{AOP-One.tex}

\part{Spring MVC}
\chapter{Spring MVC 理论}
\import{Parts/Part-MVC/Chapter-Theory}{MVC-Flow.tex}



\end{document}