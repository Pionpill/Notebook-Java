\section{Spring MVC 工作流程}

\subsection{Spring MVC 简介}

Spring MVC是Spring Framework提供的Web组件,全称是Spring Web MVC,是目前主流的实现MVC设计模式的框架,提供前端路由映射、视图解析等功能\footnote{这节参考文献: \url{https://blog.csdn.net/qq_52797170/article/details/125591705}}。

MVC是一种软件架构思想,把软件按照模型,视图,控制器来划分:
\begin{itemize}
    \item 模型(Model): 指工程中的JavaBean,用来处理数据。JavaBean分成两类:
    \begin{itemize}
        \item 实体类Bean: 专门用来存储业务数据,比如Student,User。
        \item 业务处理Bean: 指Servlet或Dao对象,专门用来处理业务逻和数据访问。
    \end{itemize}
    \item 视图(View): 指工程中的html,jsp等页面,作用是和用户进行交互,展示数据
    \item 控制(Controller): 指工程中的Servlet,作用是接收请求和响应浏览器
\end{itemize}

\begin{figure}[H]
    \small
    \centering
    \begin{tikzpicture}[scale = 1]
        \node [draw] (0) at (0,0) {浏览器};
        \node [draw] (1) at (3,0) {Controller};
        \node [draw] (2) at (3,-1) {View};
        \node [draw] (3) at (6,0) {Model(Bean)};
        \node [draw] (4) at (9,0) {DB};
        \draw[-Stealth] (0) -- (1) node [midway,above,font=\footnotesize] {请求};
        \draw[-Stealth] (2) -- (0) node [midway,below,font=\footnotesize] {响应};
        \draw[-Stealth] (1) -- (2);
        \draw[Stealth-Stealth] (1) -- (3);
        \draw[Stealth-Stealth] (3) -- (4);
    \end{tikzpicture}
    \caption{MVC 设计模式}
    \label{fig:MVC 设计模式}
\end{figure}

Spring MVC 包含以下核心组件:
\begin{itemize}
    \item DispatchServlet: 前端控制器,负责调度其他组件的执行,可以降低不同组件之间的耦合性,是整个Spring MVC的核心模块
    \item Handler: 处理器,完成具体的业务逻辑,相当于 Servlet。
    \item HandlerMapping: DispatchServlet 通过它把请求映射到不同的 Handler。
    \item HandlerInterceptor: 处理拦截器,用于进行一些拦截处理,是一个接口。
    \item HandlerExecutionChain: 处理器执行链,包括两部分内容:Handler和HandlerInterceptor。
    \item HandlerAdapter: 处理器适配器,Handler执行业务方法之前,需要进行一系列的操作。
    \item ModelAndView: 封装了模型数据和视图信息,作为Handler的处理结果,返回给DispatcherServlet。
    \item ViewResolver:视图解析器,DispatcherServlet通过它把逻辑视图解析为物理视图,最终把渲染的结果响应给客户端。
\end{itemize}

\begin{figure}[H]
    \centering
    \begin{tikzpicture}[scale = 1]
        \node [draw] (client) at (0,0) {客户端};
        \node [draw] (dispatch) at (5,0) {DispatchServlet};
        \node [draw] (map) at (12,0) {HandlerMapping};
        \node [draw] (view) at (2,-3) {ViewResolver};
        \node [draw] (adapter) at (7,-3) {HandlerAdapter};
        \node [draw] (handler) at (12,-3) {Handler};
        \begin{scope}[font = \scriptsize]
            \draw [-Stealth] ([yshift=4pt]client.east) -- ([yshift=4pt]dispatch.west) node [midway, above] {request};
            \draw [-Stealth] ([yshift=-4pt]dispatch.west) -- ([yshift=-4pt]client.east) node [midway, below] {response};
            \draw [-Stealth] ([yshift=4pt]dispatch.east) -- ([yshift=4pt]map.west) node [midway, above] {获取 Handler};
            \draw [-Stealth] ([yshift=-4pt]map.west) -- ([yshift=-4pt]dispatch.east) node [midway, below] {返回 HandlerExecutionChain};
            \draw [-Stealth] ([xshift=-20pt]dispatch.south) -- ([xshift=-4pt]view.north) node [midway, left] {解析 ModelAndView};
            \draw [-Stealth] ([xshift=4pt]view.north) -- ([xshift=-12pt]dispatch.south) node [midway, right] {返回 View};
            \draw [-Stealth] ([xshift=12pt]dispatch.south) -- ([xshift=-4pt]adapter.north) node [midway, left] {};
            \draw [-Stealth] ([xshift=4pt]adapter.north) -- ([xshift=20pt]dispatch.south) node [midway, right] {返回 ModelAndView};
            \draw [-Stealth] ([yshift=4pt]adapter.east) -- ([yshift=4pt]handler.west) node [midway, above] {执行 Handler};
            \draw [-Stealth] ([yshift=-4pt]handler.west) -- ([yshift=-4pt]adapter.east) node [midway, below] {返回 ModelAndView};
        \end{scope}
    \end{tikzpicture}
    \caption{MVC 工作流程}
    \label{fig:MVC 工作流程}
\end{figure}