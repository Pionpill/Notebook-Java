\section{IoC 的基本概念}
\subsection{Spring 框架总体结构}

Spring 本质是始终不变的,都是为了提供各种服务,以帮助我们简化基于POJO的Java应用程序开发。Spring框架为POJO提供的各种服务共同组成了Spring框架的总体结构:

\begin{figure}[H]
    \scriptsize
    \centering
    \begin{tikzpicture}[scale = 1]
        \node (core) [draw] at (0,0) {
            \begin{tabular}{cc}
                Core \\
                IoC 容器 \\
                FrameWork 工具类
            \end{tabular}
        };
        \node (aop) [draw] at (-3,3) {
            \begin{tabular}{cc}
                AOP \\
                Spring AOP \\
                AspectJ 集成
            \end{tabular}
        };
        \node (j2ee) [draw] at (0,3.5) {
            \begin{tabular}{cc}
                J2EE 服务继承 \\
                JMX,JMS\\
                JCA,EJB \\
                ......
            \end{tabular}
        };
        \node (web) [draw] at (3,3.5) {
            \begin{tabular}{cc}
                Web \\
                Spring MVC \\
                其他 Web 框架集成
            \end{tabular}
        };
        \node (dao) [draw] at (-5,6) {
            \begin{tabular}{cc}
                DAO \\
                Spring JDBC \\
                事务管理
            \end{tabular}
        };
        \node (orm) [draw] at (-2,6) {
            \begin{tabular}{cc}
                ORM \\
                Hibernate \\
                JPA \\
                .....
            \end{tabular}
        };
        \draw[-Stealth] (core) -- (aop);
        \draw[-Stealth] (core) -- (j2ee);
        \draw[-Stealth] (core) -- (web);
        \draw[-Stealth] (aop) -- (dao);
        \draw[-Stealth] (aop) -- (orm);
    \end{tikzpicture}
    \caption{Spring 框架总体结构}
    \label{fig:Spring 框架总体结构}
\end{figure}

整个框架结构最核心的部分为 IoC 容器与 AOP 框架:
\begin{itemize}
    \item \textbf{Core 核心}: 是整个框架的基础,Spring 为我们提供了一个 IoC 容器,用于帮助我们以\textbf{依赖注入}的方式管理对象之间的依赖关系。除此之外,Core核心模块中还包括框架内部使用的各种工具类。
    \item \textbf{AOP}: 提供了一个轻便但功能强大的AOP框架,让我们可以以AOP的形式增强各POJO(Plain Old Java Object,简单Java对象)的能力。
\end{itemize}

其余的 DAO, JDBC 是 AOP 提供的数据库操作,事务管理服务。在 IoC 容器的基础上可以使用各种 J2EE 服务和 Web 框架。

\subsection{IoC 理念}

IoC 全称 Inversion of Control,中文通常翻译为“控制反转”。IoC 中有一个重要的机制叫依赖注入 DI(Dependency Injection)。

\fbox{
    \parbox{0.87\textwidth}{
        \begin{dispute}
            原书将 IoC 与DI等同看待,这并不合理,本文抛弃了这种看法。
        \end{dispute}
    }
}

要理解依赖注入,首选需要理解依赖的含义,看下面一段代码:

\begin{Java}
public class Provider {
    private Listener newsListener = new NewsListener();      // 依赖对象
    private Persister newsPersistener = new NewsPersistener();   // 依赖对象
    public void getAndPersistNews () {
        ...
    }
}
\end{Java}

其中, Provider 需要依赖 newsListener 来抓取内容,并依赖 newPersistener 存储抓取的内容。这两个类被称为 Provider 的依赖类。

如果我们依赖于某个类或服务,最简单而有效的方式就是直接在类的构造函数中 new 相应的依赖类。这个过程是我们自己主动去获取依赖的对象。

可是回头想想,我们自己每次用到什么依赖对象都要主动地去获取,这是否真的必要?我们最终所要做的,其实就是直接调用依赖对象所提供的某项服务而已。只要用到这个依赖对象的时候,它能够准备就绪,我们完全可以不管这个对象是自己找来的还是别人送过来的。如果有人能够在我们需要时将某个依赖对象送过来,为什么还要大费周折地自己去折腾?

实际上,IoC就是为了帮助我们避免之前的“大费周折”,而提供了更加轻松简洁的方式。它的反转,就反转在让你从原来的事必躬亲,转变为现在的享受服务。简单点儿说,IoC的理念就是,让别人为你服务!

\begin{figure}[H]
    \small
    \centering
    \begin{tikzpicture}[scale = 1]
        \node (provider) [draw, ellipse] at (0,0) {IoC Service Provider};
        \node (object1) [draw] at (-3,-2) {被注入的对象};
        \node (object2) [draw] at (3,-2) {被依赖的对象};
        \draw [-Stealth] (provider) -| (object1);
        \draw [-Stealth] (provider) -| (object2);
        \draw [-Stealth,dashed] (object1) -- (object2) node [above, midway] {依赖于};
    \end{tikzpicture}
    \caption{IoC 的角色}
    \label{fig:IoC 的角色}
\end{figure}

通常情况下,被注入对象会直接依赖于被依赖对象。但是,在IoC的场景中,二者之间通过IoC Service Provider来打交道,所有的被注入对象和依赖对象现在由IoC Service Provider统一管理。被注入对象需要什么,直接跟IoC Service Provider招呼一声,后者就会把相应的被依赖对象注入到被注入对象中,从而达到IoC Service Provider为被注入对象服务的目的。IoC Service Provider在这里就是通常的IoC容器所充当的角色。从被注入对象的角度看,与之前直接寻求依赖对象相比,依赖对象的取得方式发生了反转,控制也从被注入对象转到了IoC Service Provider那里。

IoC 容器负责创建的对象,初始化等一系列工作,被创建或被管理的对象在 IoC 容器中统称为 Bean。

\subsection{依赖注入的方式}

原书给了一个形象的比喻:当你来到酒吧,想要喝杯啤酒的时候,通常会直接招呼服务生,让他为你送来一杯清凉解渴的啤酒。
\begin{itemize}
    \item 如果你是酒吧的常客,或许你刚坐好,服务生已经将你最常喝的啤酒放到了你面前;
    \item 如果你是初次或偶尔光顾,也许你坐下之后还要招呼服务生;
    \item 还有一种可能,你根本就不知道哪个牌子是哪个牌子,这时,你只能打手势或干脆画出商标图来告诉服务生你到底想要什么了吧!
\end{itemize}

这很形象地介绍了三种依赖注入的方式:
\begin{itemize}
    \item 构造方法注入: constructor injection
    \item setter 方法注入: setter injection
    \item 接口注入: interface injection
\end{itemize}

\subsubsection{构造方法注入}

顾名思义,通过构造方法中声明依赖对象的参数列表,让外部 (通常是 IoC 容器) 知道它需要哪些依赖对象。

\begin{Java}
public FXNewsProvider(IFXNewsListener newsListner,IFXNewsPersister newsPersister) {   
    this.newsListener   = newsListner; 
    this.newPersistener = newsPersister;
}
\end{Java}

IoC Service Provider会检查被注入对象的构造方法,取得它所需要的依赖对象列表,进而为其注入相应的对象。同一个对象是不可能被构造两次的,因此,被注入对象的构造乃至其整个生命周期,应该是由IoC Service Provider来管理的。

构造方法注入方式比较直观,对象被构造完成后,即进入就绪状态,可以马上使用。这就好比你刚进酒吧的门,服务生已经将你喜欢的啤酒摆上了桌面一样。坐下就可马上享受一份清凉与惬意。

\subsubsection{setter 方法注入}

对于JavaBean对象来说,通常会通过setXXX()和getXXX()方法来访问对应属性。这些setXXX()方法统称为setter方法,getXXX()当然就称为getter方法。通过setter方法,可以更改相应的对象属性,通过getter方法,可以获得相应属性的状态。所以,当前对象只要为其依赖对象所对应的属性添加setter方法,就可以通过setter方法将相应的依赖对象设置到被注入对象中。

\begin{Java}
public class FXNewsProvider {
    private IFXNewsListener  newsListener;
    private IFXNewsPersister newPersistener;
    public IFXNewsListener getNewsListener() {     
        return  newsListener;    
    }    
    public void setNewsListener(IFXNewsListener newsListener) {     
        this.newsListener = newsListener;   
    }    
    public IFXNewsPersister getNewPersistener() {     
        return  newPersistener;    
    }    
    public void setNewPersistener(IFXNewsPersister newPersistener) {     
        this.newPersistener = newPersistener;   
    }
}
\end{Java}

这样,外界就可以通过调用setNewsListener和setNewPersistener方法为FXNewsProvider对象注入所依赖的对象了。

setter方法注入虽不像构造方法注入那样,让对象构造完成后即可使用,但相对来说更宽松一些,可以在对象构造完成后再注入。这就好比你可以到酒吧坐下后再决定要点什么啤酒,可以要百威,也可以要大雪,随意性比较强。如果你不急着喝,这种方式当然是最适合你的。

\subsubsection{接口注入}

相对于前两种注入方式来说,接口注入没有那么简单明了。被注入对象如果想要IoC  Service  Provider为其注入依赖对象,就必须实现某个接口。这个接口提供一个方法,用来为其注入依赖对象。IoC Service Provider最终通过这些接口来了解应该为被注入对象注入什么依赖对象。

接口注入的方式比较死板,这里不做过多说明。

\subsubsection{三种注入方式的比较}

\begin{itemize}
    \item \textbf{接口注入}: 接口注入是现在不甚提倡的一种方式,基本处于“退役状态”。因为它强制被注入对象实现不必要的接口,带有侵入性。
    \item \textbf{构造方法注入}: 这种注入方式的优点就是,对象在构造完成之后,即已进入就绪状态,可以马上使用。缺点如下: 
        \begin{itemize}
            \item 当依赖对象比较多的时候,构造方法的参数列表会比较长。而通过反射构造对象的时候,对相同类型的参数的处理会比较困难,维护和使用上也比较麻烦。
            \item 构造方法无法被继承,无法设置默认值。对于非必须的依赖处理,可能需要引入多个构造方法,而参数数量的变动可能造成维护上的不便。
        \end{itemize}
    \item \textbf{setter 方法注入}: 因为方法可以命名,所以setter方法注入在描述性上要比构造方法注入好一些。另外,setter方法可以被继承,允许设置默认值,而且有良好的IDE支持。缺点当然就是对象无法在构造完成后马上进入就绪状态。
\end{itemize}

\newpage