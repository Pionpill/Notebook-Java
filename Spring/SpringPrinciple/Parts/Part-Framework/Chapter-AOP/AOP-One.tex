\section{Spring AOP 一世}

在动态代理和CGLIB的支持下,SpringAOP框架的实现经过了两代。从Spring的AOP框架第一次发布,到Spring2.0发布之前的AOP实现,是Spring第一代AOP实现。Spring2.0发布后的AOP实现是第二代。

不过划分归划分,SpringAOP的底层实现机制却一直没变。唯一改变的,是各种AOP概念实体的表现形式以及Spring AOP的使用方式。

\subsection{Spring AOP 中的 Joinpoint}

AOP 的 JoinPoint 可以有许多种类型,如构造方法调用,字段的设置及获取,方法调用,方法执行等。但在 SpringAOP 中,仅支持方法级别的 Joinpoint。但在实际的开发过程中,这已经可以满足 80\% 的需求了。

Sring AOP 这样做的原因有以下几点:
\begin{itemize}
    \item Spring AOP 提供一个强大而简单的 AOP 框架,但不会因为追求强大而变得臃肿。
    \item 对于类中属性(Field)级别的 Joinpoint,如果提供这个级别的拦截支持,那么久破坏了面向对象的封装,而且完全可以通过 setter 和 getter 方法的拦截达到同样的目的。
    \item 如果应用要求非常特殊,不妨求助于其他 AOP 产品。如 AspectJ。
\end{itemize}

\subsection{Spring AOP 中的 PointCut}

Spring中以接口定义 \texttt{Pointcut} 作为其AOP框架中所有Pointcut的最顶层抽象,该接口定义了两个方法用来帮助捕捉系统中的相应Joinpoint,并提供了一个 \texttt{TruePointcut} 类型实例。

如果 \texttt{Pointcut} 类型为 \texttt{TruePointcut} ,默认会对系统中的所有对象,以及对象上所有被支持的Joinpoint进行匹配。

\texttt{Pointcut} 接口定义如下所示:

\begin{Java}
public interface Pointcut {
    ClassFilter getClassFilter();
    MethodMatcher getMethodMatcher();
    Pointcut TRUE = TruePointcut.INSTANCE;
}      
\end{Java}

其中 \texttt{ClassFilter} 和 \texttt{MethodMatcher} 分别用于匹配将执行织入操作的对象以及相应的方法。之所以将这两个分开,是为了重用不同级别的匹配定义。并且可以在不同的级别或者相同的级别上进行组合操作,或者强制让某个子类覆写相应的方法定义等。

\texttt{ClassFilter} 接口的作用是对\texttt{Joinpoint} 所处的对象进行 \texttt{Class} 级别的类型匹配,定义如下:

\begin{Java}
public interface ClassFilter {
    boolean matches(Class clazz);
    ClassFilter TRUE = TrueClassFilter.INSTANCE;
}
\end{Java}

当织入的目标对象的Class类型与Pointcut所规定的类型相符时,matches方法将会返回true,否则,返回false,即意味着不会对这个类型的目标对象进行织入操作。比如,如果我们仅希望对系统中Foo类型的类执行织入,则可以如下这样定义ClassFilter:

\begin{Java}
public boolean matches(Class clazz) {
    return Foo.class.equals(clazz);
}
\end{Java}

相对于 \texttt{ClassFilter} 的简单定义,\texttt{MethodMatcher} 则要复杂得多。毕竟,Spring主要支持的就是方法级别的拦截——“重头戏”可不能单薄啊!\texttt{MethodMatcher}定义如下:

\begin{Java}
public interface MethodMatcher {
    boolean matches(Method method, Class targetClass);
    boolean isRuntime();
    boolean matches(Method method, Class targetClass, Object[] args);
    MethodMatcher TRUE = TrueMethodMathcer.INSTANCE;
}      
\end{Java}

\texttt{MethodMatcher} 定义了两个 \texttt{matches} 方法,而这两个方法的分界线就是 \texttt{isRuntime} 方法。在对对象具体方法进行拦截的时候,可以忽略每次方法执行的时候调用者传入的参数,也可以每次都检查这些方法调用参数,以强化拦截条件。假设对以下方法进行拦截:

\begin{Java}
public boolean login(String username, String password);
\end{Java}

如果只想在login方法之前插入计数功能,那么login方法的参数对于Joinpoint捕捉就是可以忽略的。而如果想在用户登录的时候对某个用户做单独处理,如不让其登录或者给予特殊权限,那么这个方法的参数就是在匹配Joinpoint的时候必须要考虑的。

\begin{itemize}
    \item 前一种情况下,\texttt{isRuntime} 返回 \texttt{false},表示不会考虑具体 \texttt{Joinpoint} 的方法参数,这种类型的 \texttt{MethodMatcher} 称之为 \texttt{StaticMethodMatcher} 。
    
    因为不用每次都检查参数,那么对于同样类型的方法匹配结果,就可以在框架内部缓存以提高性能。\texttt{isRuntime} 方法返回 \texttt{false} 表明当前的 \texttt{MethodMatcher} 为 \texttt{StaticMe thodMatcher} 的时候,只有\texttt{boolean matches(Method method, Class targetClass);}方法将被执行,它的匹配结果将会成为其所属的 \texttt{Pointcut} 主要依据。
    \item 当 \texttt{isRuntime} 方法返回 \texttt{true} 时,表明该 \texttt{MethodMatcher} 将会每次都对方法调用的参数进行匹配检查,这种类型的 \texttt{MethodMatcher} 称之为 \texttt{DynamicMethodMatcher} 。
    
    因为每次都要对方法参数进行检查,无法对匹配的结果进行缓存,所以,匹配效率相对于\texttt{staticMethodMatcher}来说要差。而且大部分情况下,\texttt{staticMethodMatcher}已经可以满足需要,最好避免使用\texttt{DynamicMethodMatcher}类型。
\end{itemize}

\subsection{Spring AOP 中的 Advice}

Spring中各种Advice类型实现与AOPAlliance中标准接口之间的关系如下图:

\begin{figure}[H]
    \scriptsize
    \centering
    \begin{tikzpicture}[scale = 1]
        \begin{interface}[text width=2cm]{Advice}{0,0}
        \end{interface}
        \begin{interface}[text width=2cm]{AfterAdvice}{-3,-2}
            \inherit{Advice}
        \end{interface}
        \begin{interface}[text width=2cm]{BeforeAdvice}{0,-2}
            \inherit{Advice}
        \end{interface}
        \begin{interface}[text width=2cm]{Interecptor}{3,-2}
            \inherit{Advice}
        \end{interface}
        \begin{interface}[text width=2.5cm]{AfterReturningAdvice}{-6.5,-4}
            \inherit{AfterAdvice}
        \end{interface}
        \begin{interface}[text width=2cm]{ThrowsAdvice}{-3,-4}
            \inherit{AfterAdvice}
        \end{interface}
        \begin{interface}[text width=2.5cm]{MethodBeforeAdvice}{0,-4}
            \inherit{BeforeAdvice}
        \end{interface}
        \begin{interface}[text width=2.5cm]{MethodInterceptor}{3,-4}
            \inherit{Interecptor}
        \end{interface}
    \end{tikzpicture}
    \caption{Spring 中的 Advice 缩略图}
    \label{fig:Spring 中的 Advice 缩略图}
\end{figure}

Advice 实现了被织入到 Pointcut 规定的 Joinpoint 处的横切逻辑。

在Spring中,Advice按照其自身实例能否在目标对象类的所有实例中共享这一标准,可以划分为两大类,即\texttt{per-class}类型的Advice和\texttt{per-instance}类型的Advice。

\paragraph*{per-class 类型的 Advice}该类型的Advice的实例可以在目标对象类的所有实例之间共享。这种类型的Advice通常只是提供方法拦截的功能,不会为目标对象类保存任何状态或者添加新的特性。上图中没有列出的 \texttt{Introduction} 类型的Advice不属于 \texttt{perclass} 类型的Advice之外,图中的所有Advice均属此列。

\paragraph*{per-instance 类型的 Advice} 该类型的Advice不会在目标类所有对象实例之间共享,而是会为不同的实例对象保存它们各自的状态以及相关逻辑。在SpringAOP中, \texttt{Introduction} 就是唯一的一种 \texttt{per-instance} 型Advice。

\subsection{Spring AOP 中的 Aspect}

Advisor代表Spring中的Aspect,但是,与正常的Aspect不同,Advisor通常只持有一个Pointcut和一个Advice。而理论上,Aspect定义中可以有多个Pointcut和多个Advice,所以,我们可以认为Advisor是一种特殊的Aspect。

我们可以将Advisor简单划分为两个分支,一个分支以 \texttt{PointcutAdvisor} 为首,另一个分支则以 \texttt{IntroductionAdvisor} 为头儿。

系统中只存在单一的横切关注点的情况很少,大多数时候,都会有多个横切关注点需要处理,那么,系统实现中就会有多个Advisor存在。当其中的某些Advisor的Pointcut匹配了同一个Joinpoint的时候,就会在这同一个Joinpoint处执行多个Advice的横切逻辑。

如果这些Advisor所关联的Advice之间没有很强的优先级依赖关系,那么谁先执行,谁后执行都不会造成任何影响。而一旦这几个需要在同一Joinpoint处执行的Advice逻辑存在优先顺序依赖的话,就需要我们来干预了,否则,系统的行为就会偏离我们的预想。

Spring在处理同一Joinpoint处的多个Advisor的时候,实际上会按照指定的顺序和优先级来执行它们,顺序号决定优先级,顺序号越小,优先级越高,优先级排在前面的,将被优先执行。我们可以从0或者1开始指定,因为小于0的顺序号原则上由SpringAOP框架内部使用。默认情况下,如果我们不明确指定各个Advisor的执行顺序,那么Spring会按照它们的声明顺序来应用它们,最先声明的顺序号最小但优先级最大,其次次之。

\subsection{Spring AOP 织入}

Spring AOP 使用 \texttt{ProxyFactory} 作为织入器。\texttt{ProxyFactory} 并非SpringAOP中唯一可用的织入器,而是最基本的一个织入器实现。 \texttt{ProxyFactory} 织入非常简单:

\begin{Java}
ProxyFactory weaver = new ProxyFactory(yourTargetObject);
// 或者
// ProxyFactory weaver = new ProxyFactory();
// weaver.setTarget(task);

Advisor advisor = ...;
weaver.addAdvisor(advisor);
Object proxyObject = weaver.getProxy();
\end{Java}

\newpage