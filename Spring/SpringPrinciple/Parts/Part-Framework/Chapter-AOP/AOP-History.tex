\section{Spring AOP 历史}

在动态代理和CGLIB的支持下,SpringAOP框架的实现经过了两代。从Spring的AOP框架第一次发布,到Spring2.0发布之前的AOP实现,是Spring第一-代AOP实现。Spring2.0发布后的AOP实现是第二代。

不过划分归划分,SpringAOP的底层实现机制却一直没变。唯一改变的,是各种AOP概念实体的表现形式以及Spring AOP的使用方式。

\subsection{Spring AOP 一世}

\subsubsection{Spring AOP 中的 Joinpoint}

AOP 的 JoinPoint 可以有许多种类型,如构造方法调用,字段的设置及获取,方法调用,方法执行等。但在 SpringAOP 中,仅支持方法级别的 Joinpoint。但在实际的开发过程中,这已经可以满足 80\% 的需求了。

Sring AOP 这样做的原因有以下几点:
\begin{itemize}
    \item Spring AOP 提供一个强大而简单的 AOP 框架,但不会因为追求强大而变得臃肿。
    \item 对于类中属性(Field)级别的 Joinpoint,如果提供这个级别的拦截支持,那么久破坏了面向对象的封装,而且完全可以通过 setter 和 getter 方法的拦截达到同样的目的。
    \item 如果应用要求非常特殊,不妨求助于其他 AOP 产品。如 AspectJ。
\end{itemize}

\subsubsection{Spring AOP 中的 PointCut}

Spring中以接口定义 \texttt{Pointcut} 作为其AOP框架中所有Pointcut的最顶层抽象,该接口定义了两个方法用来帮助捕捉系统中的相应Joinpoint,并提供了一个 \texttt{TruePointcut} 类型实例。

如果 \texttt{Pointcut} 类型为 \texttt{TruePointcut} ,默认会对系统中的所有对象,以及对象上所有被支持的Joinpoint进行匹配。

\texttt{Pointcut} 接口定义如下所示:

\begin{Java}
public interface Pointcut {
    ClassFilter getClassFilter();
    MethodMatcher getMethodMatcher();
    Pointcut TRUE = TruePointcut.INSTANCE;
}      
\end{Java}

其中 \texttt{ClassFilter} 和 \texttt{MethodMatcher} 分别用于匹配将执行织入操作的对象以及相应的方法。之所以将这两个分开,是为了重用不同级别的匹配定义。并且可以在不同的级别或者相同的级别上进行组合操作,或者强制让某个子类覆写相应的方法定义等。

\texttt{ClassFilter} 接口的作用是对\texttt{Joinpoint} 所处的对象进行 \texttt{Class} 级别的类型匹配,定义如下:

\begin{Java}
public interface ClassFilter {
    boolean matches(Class clazz);
    ClassFilter TRUE = 
}
\end{Java}










\newpage