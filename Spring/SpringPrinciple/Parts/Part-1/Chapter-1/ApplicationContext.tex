\section{ApplicationContext}

ApplicationContext除了拥有BeanFactory支持的所有功能之外,还进一步扩展了基本容器的功能,包括BeanFactoryPostProcessor、BeanPostProcessor以及其他特殊类型bean的自动识别、容器启动后bean实例的自动初始化、国际化的信息支持、容器内事件发布等。

\subsection{统一资源加载策略}

\subsubsection{Spring 中的 Resource}

Spring框架内部使用 Resource 接口作为所有资源的抽象和访问接口。

Resource接口可以根据资源的不同类型,或者资源所处的不同场合,给出相应的具体实现。Spring框架在这个理念的基础上,提供了一些实现类:

\begin{itemize}
    \item \textbf{ByteArrayResource}: 将字节(byte)数组提供的数据作为一种资源进行封装,如果通过InputStream形式访问该类型的资源,该实现会根据字节数组的数据,构造相应的ByteArrayInputStream并返回。
    \item \textbf{ClassPathResource}: 该实现从Java应用程序的ClassPath中加载具体资源并进行封装,可以使用指定的类加载器(ClassLoader)或者给定的类进行资源加载。
    \item \textbf{FileSystemResource}: 对java.io.File类型的封装,所以,我们可以以文件或者URL的形式对该类型资源进行访问,只要能跟File打的交道,基本上跟FileSystemResource也可以。
    \item \textbf{UrlResource}: 通过java.net.URL进行的具体资源查找定位的实现类,内部委派URL进行具体的资源操作。
    \item \textbf{InputStreamResource}: 将给定的InputStream视为一种资源的Resource实现类,较为少用。
\end{itemize}

