\section{Wrapper 包装类}

String 对八个基本数据类型提供了对应的包装类,根据类型我分为了以下几种:

\begin{itemize}
    \item 整型: Byte, Short, Integer, Long
    \item 浮点型: Float, Double
    \item 布尔: Boolean
    \item 字符: Character
\end{itemize}

其中整型方法类似,布尔型没什么好讲的。包装类是装饰器模式的一种应用,部分人也将 Wrapper 等价于装饰器模式。

\subsection{Integer 数值型包装类}

\begin{Java}
public final class Integer extends Number 
    implements Comparable<Integer>, Constable, ConstantDesc {}
\end{Java}

\subsubsection{构造对象}

最简单的,使用构造函数,Integer 的构造函数有两个:

\begin{Java}
@Deprecated(since="9", forRemoval = true)
public Integer(int value) {
    this.value = value;
}

@Deprecated(since="9", forRemoval = true)
public Integer(String s) throws NumberFormatException {
    this.value = parseInt(s, 10);
}
\end{Java}

注意,这两个构造函数虽然可以用,但都被舍弃了,推荐使用静态的 ValueOf 方法。类似的有两类静态方法:

\begin{Java}
@IntrinsicCandidate
public static Integer valueOf(int i) {
    if (i >= IntegerCache.low && i <= IntegerCache.high)
        return IntegerCache.cache[i + (-IntegerCache.low)];
    return new Integer(i);
}

public static Integer valueOf(String s, int radix) throws NumberFormatException {
    return Integer.valueOf(parseInt(s,radix));
}

public static Integer valueOf(String s) throws NumberFormatException {
    return Integer.valueOf(parseInt(s, 10));
}
\end{Java}

其中传入 int 型的方法做了一些优化,缓存了-128—127的Integer值,如遇到[-128,127]范围的值需要转换为Integer时会直接从IntegerCache中获取\footnote{IntegerCache 拓展: \url{https://blog.csdn.net/cockroach02/article/details/88857551}}。IntegerCache 存储在 JVM 方法区运行时常量池中。另外 @IntrinsicCandidate 注解表示在虚拟机中做了优化。

传入 String 类型的方法,除了可增加一个参数: 基数外,调用了 parseInt 方法。 parseInt 的工作很简单:检查字符串是否合法并进行转换。

\subsubsection{类型转换}

首先是 toString 函数,Integer 提供了各种各样的转换为字符串的方法,这里不一一说明:

\begin{Java}
// 全能款:转为为各种 String 类型
public static String toString(int i, int radix) { ... }
public static String toUnsignedString(int i, int radix) { ... }
// 便携款:相当于制定了 radix
public static String toHexString(int i) { ... }
......
\end{Java}

其次是 parse 函数,用于将各种字符串或字符序列进行解析,注意返回的是 int 类型的值:

\begin{Java}
// 解析 String
public static int parseInt(String s, int radix) { ... }
public static int parseUnsignedInt(String s, int radix) { ... }
// 解析 序列
public static int parseInt(CharSequence s, int beginIndex, int endIndex, int radix) throws NumberFormatException {}
\end{Java}

最后是 Value 函数,用于转换为其他类型的整型/浮点值:

\begin{Java}
// 本质就是强制类型转换,其他类型一样
public long longValue() { return (long)value; }
\end{Java}

\subsubsection{数学运算}

Integer 提供了基本的数学运算函数,本质上和直接使用符号运算效果相同:

\begin{Java}
// 比较运算
public static int compare(int x, int y) {
    return (x < y) ? -1 : ((x == y) ? 0 : 1);
}
// 除法运算
public static int divideUnsigned(int dividend, int divisor) {
    return (int)(toUnsignedLong(dividend) / toUnsignedLong(divisor));
}
// 取余运算
public static int remainderUnsigned(int dividend, int divisor) {
    return (int)(toUnsignedLong(dividend) % toUnsignedLong(divisor));
}
\end{Java}

另外提供了一些好用位运算方法:

\begin{Java}
// 最高的一位,理解为比数字小的最大 2 的整数倍,例如 17 -> 16  7-> 4
public static int highestOneBit(int i) {
    return i & (MIN_VALUE >>> numberOfLeadingZeros(i));
}
// 最低的一位,解释起来比较麻烦,自行理解
public static int lowestOneBit(int i) { 
    return i & -i;
}
// 前置 0 的个数,注意占 32 bit
@IntrinsicCandidate
public static int numberOfLeadingZeros(int i) { ... }
// 后置 0 的个数
@IntrinsicCandidate
public static int numberOfTrailingZeros(int i) { ... }
// 二进制中 1 的个数
@IntrinsicCandidate
public static int bitCount(int i) { ... }
\end{Java}

\subsubsection{Float 浮点型}

Float 大部分方法和 Integer 类似,这里只说点特别的:

\begin{Java}
// 判断是否是 NaN (Not a Number)
public static boolean isNaN(float v) {
    return (v != v);
}
// 判断是否无穷
public static boolean isInfinite(float v) { ... }
\end{Java}

\subsection{Character 字符型包装类}

\subsubsection{常用函数}

判断型常用函数:

\begin{Java}
// 判断大小写字符
public static boolean isLowerCase(char ch) { ... }
public static boolean isUpperCase(char ch) { ... }
// 判断数字还是字符
public static boolean isDigit(char ch) { ... }
public static boolean isLetter(char ch) { ... }
public static boolean isLetterOrDigit(char ch) { ... }
// 判断是否为空
public static boolean isWhitespace(char ch) { ... }
\end{Java}

转换型常用函数

\begin{Java}
public static char toLowerCase(char ch) { ... }
public static char toUpperCase(char ch) { ... }
\end{Java}

\subsection{自动装箱与拆箱机制}

包装类提供的语法糖自然是自动装箱与拆箱机制,下面是编译器自动帮我们完成的事情:

\begin{Java}
// ========== 自动装箱 ==========
Integer a = 1;
Integer a = Integer.valueOf(1); // 实际过程
// ========== 自动拆箱 ==========
int b = a;
int b = a.intValue();           // 实际过程
int c = a + a;
int c = a.intValue() + a.intValue();    // 实际过程
\end{Java}

\newpage