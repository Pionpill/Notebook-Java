\section{Object 基类}

Object 是所有类的基类,这节将详细说明它的所有成员。

\subsection{类与对象}

\noindent\textbf{构造函数}
\begin{Java}
@IntrinsicCandidate
public Object() {}
\end{Java}

其中 @IntrinsicCandidate 表示在虚拟机中会有一套高效的实现方式。其他没什么。

\noindent\textbf{反射相关}
\begin{Java}
@IntrinsicCandidate
public final native Class<?> getClass();
\end{Java}

该方法用于获取对象的类型信息。即在 JVM 中获取方法区中对应的类型信息。此外这里有一个 native 关键字,表示对应的方法无法通过 Java 语言实现,而应该让 JVM 调用 JNI(Java Native Interface)通过更底层的 C/C++ 语言实现。

\subsection{常见方法}

\noindent\textbf{哈希码}\footnote{参考: \url{https://blog.csdn.net/SEU_Calvin/article/details/52094115}}

\begin{Java}
@IntrinsicCandidate
public native int hashCode();
\end{Java}

哈希码与 equals 方法有如下联系:
\begin{itemize}
    \item 如果两个对象相同, equals 方法一定返回 true,这两个对象的HashCode一定相同;
    \item 两个对象的HashCode相同,并不一定表示两个对象就相同,即equals()不一定为true,只能说明这两个对象在一个散列存储结构中。
    \item 如果对象的equals方法被重写,那么对象的HashCode也尽量重写。
\end{itemize}

哈希码的作用主要是快速查找,常常被用在容器(Collection)中。哈希算法也称为散列算法,是将数据依特定算法直接指定到一个地址上。这样一来,当集合要添加新的元素时,先调用这个元素的HashCode方法,就一下子能定位到它应该放置的物理位置上。

\begin{itemize}
    \item 如果这个位置上没有元素,它就可以直接存储在这个位置上;
    \item 如果这个位置上已经有元素了,就调用它的equals方法与新元素进行比较,相同的话就不存了;
    \item 不相同的话,也就是发生了Hash key相同导致冲突的情况,那么就在这个Hash key的地方产生一个链表,将所有产生相同HashCode的对象放到这个单链表上去,串在一起。这样一来实际调用equals方法的次数就大大降低了,几乎只需要一两次。 
\end{itemize}

\noindent\textbf{equals}

\begin{Java}
public boolean equals(Object obj) {
    return (this == obj);
}
\end{Java}

Object 本身的 equals 方法没什么好说的,就是通过 == 比较地址是否相同。但像 String 这样的绝大部分子类都会重写该方法,让其效果为: 比较对象内容是否相等。

比较地址实际上是比较的存在 JVM 栈中的 reference 是否相同。

\noindent\textbf{clone}

\begin{Java}
@IntrinsicCandidate
protected native Object clone() throws CloneNotSupportedException;
\end{Java}

JDK 文档中堆 clone() 方法有如下约定(前两条必须遵守):
\begin{itemize}
    \item \texttt{x.clone != x;} 克隆对象与原对象不是一个对象。
    \item \texttt{x.clone().getClass() == x.getClass();} 克隆的是同一类型的对象。
    \item \texttt{x.clone().equals(x) == true} 如果x.equals()方法定义恰当的话
\end{itemize}

一般情况下,子类并不需要实现 \texttt{clone()} 方法,Object 的 \texttt{clone()} 方法将返回一个浅拷贝(由于是 native 的,非常快),对于简单对象这就足够了。但如果需要实现深拷贝,需要子类实现 \texttt{Cloneable()} 接口,并重写 \texttt{clone()} 方法。

\noindent\textbf{toString}

\begin{Java}
public String toString() {
    return getClass().getName() + "@" + Integer.toHexString(hashCode());
}
\end{Java}

返回对象的字符串表达形式,没啥好说的,\texttt{print()} 方法会自动调用。

\subsection{多线程}

\noindent\textbf{唤醒线程}

\begin{Java}
@IntrinsicCandidate
public final native void notify();

@IntrinsicCandidate
public final native void notifyAll();
\end{Java}

这两个就一个不同,\texttt{notify()} 随机唤醒一个线程,\texttt{notifyAll()} 广播机制,让其他线程抢占资源。

\noindent\textbf{线程等待}

\begin{Java}
public final void wait() throws InterruptedException { wait(0L); }

public final native void wait(long timeoutMillis) throws InterruptedException;

public final void wait(long timeoutMillis, int nanos) throws InterruptedException {
    if (timeoutMillis < 0)
        throw new IllegalArgumentException("timeoutMillis value is negative");
    if (nanos < 0 || nanos > 999999)
        throw new IllegalArgumentException( "nanosecond timeout value out of range");
    if (nanos > 0 && timeoutMillis < Long.MAX_VALUE)
        timeoutMillis++;
    wait(timeoutMillis);
}
\end{Java}

第一个 \texttt{wait()} 让线程等待,直到被唤醒,不被唤醒不启动。其余两个, timeoutMillis 表示给定的时间,如果到了给定的时间还没被唤醒,会自动醒来尝试抢占资源(timeoutMillis 为 0 表示不会自动醒)。其实本质上都是去调用第二个 native 方法。

其中第三个有个 \texttt{nanos} 参数,表示格外的时间,单位为纳秒。

\newpage