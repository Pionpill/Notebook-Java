\section{Reflect 反射}

JAVA反射机制是在运行状态中,对于任意一个类,都能够知道这个类的所有属性和方法; 对于任意一个对象,都能够调用它的任意一个方法和属性; 这种动态获取的信息以及动态调用对象的方法的功能称为java语言的反射机制\footnote{本节参考: \url{https://blog.csdn.net/lililuni/article/details/83449088}}。

实际上,我们创建的每一个类也都是对象,即类本身是java.lang.Class类的实例对象。这个实例对象称之为类对象,也就是Class对象。

放射机制在框架中使用的非常普遍,理解反射机制有助于理解例如 Spring 等框架。

\subsection{Class 对象}

在 \texttt{Object} 源码中有这样一段描述: Class {@code Object} is the root of the class hierarchy. 既然类本身是java.lang.Class类的实例对象,那么类自身也就可以调用 Class 的一些方法。

\texttt{Class} 类的声明如下:

\begin{Java}
public final class Class<T> implements java.io.Serializable, GenericDeclaration, Type, AnnotatedElement, TypeDescriptor.OfField<Class<?>>, Constable
\end{Java}

\begin{itemize}
    \item \texttt{Class} 类的实例对象表示正在运行的 Java 应用程序中的类和接口。也就是jvm中有很多的实例,每个类都有唯一的\texttt{Class}对象。
    \item \texttt{Class} 类没有公共构造方法。\texttt{Class} 对象是在加载类时由 Java 虚拟机自动构造的。也就是说我们不需要创建,JVM已经帮我们创建了。
    \item \texttt{Class} 对象用于提供类本身的信息,比如有几种构造方法, 有多少属性,有哪些普通方法。
\end{itemize}

\texttt{CLass} 本身是有``构造函数''的,只不过我们无法调用,而由 JVM 调用:

\begin{Java}
// Private constructor. Only the Java Virtual Machine creates Class objects.
// This constructor is not used and prevents the default constructor being generated.
private Class(ClassLoader loader, Class<?> arrayComponentType) {
    classLoader = loader;
    componentType = arrayComponentType;
}
\end{Java}

获取类对象的方式有三种(假设有一个 Hero 类):
\begin{itemize}
    \item 静态方法(常用): \texttt{Class.forName("com.xxx.Hero")}
    \item 公有属性: \texttt{Hero.class}
    \item 对象方法: \texttt{new Hero().getClass()}
\end{itemize}

其中 \texttt{forName} 的源码如下:

\begin{Java}
@CallerSensitive
public static Class<?> forName(String className) throws ClassNotFoundException { 
    // native 操作 
}
\end{Java}

这里有个注解 @CallerSensitive 代表这些方法非常危险,不希望开发者调用,并会在 jvm 级别进行检查。\texttt{forName} 的具体操作是使用 \texttt{native} 修饰的,我们无从得知。

为了第一种方法危险但还是常用呢? 当然是全靠同行承托,第二种方法需要导入类的包,依赖性太强。第三种方法都已经有了对象还要反射干嘛。不过一般开发者不需要调用这些方法,做框架的人才回去考虑。

反射机制不是很常用,因此这节不一一讲源码,只对常用的操作做一些解释。

\subsection{Constructor 与创建新对象}

利用反射机制,一般会需要获取对象实例,获取对象实例的方式则是通过构造函数创建对象实例。

\begin{Java}
Class class = Class.forName("com.xxx.Hero");    // 获取 Class 对象
Constructor con = class.getConstructor();       // 通过 Class 对象获取构造函数
\end{Java}

\texttt{Class} 类中有两种获取构造函数对象的方法,分别为批量获取构造函数对象,获取有参的构造函数对象:

\begin{Java}
// 无参构造函数对象数组
@CallerSensitive
public Constructor<?>[] getConstructors() throws SecurityException
// 有参构造函数对象,注意参数是类型
@CallerSensitive
public Constructor<T> getConstructor(Class<?>... parameterTypes)
    throws NoSuchMethodException, SecurityException
\end{Java}

同以上两种方法都只能获得公有的构造方法,想获得私有构造方法只需要将 \texttt{getConstructor} 替换为 \texttt{getDeclaredConstructor}。下文也类似,加上 \texttt{Declared} 基本都代表跳过安全检查,获取私有成员。

这样我们就获取了 \texttt{Constructor} 对象。
\begin{Java}
public final class Constructor<T> extends Executable
\end{Java}

在 \texttt{Constructor} 类中有功能性方法如下:

\begin{Java}
// 暴力访问,如果参数为 true,忽视访问修饰符
@Override
@CallerSensitive
public void setAccessible(boolean flag)
// 创建实例对象
@CallerSensitive
@ForceInline
public T newInstance(Object ... initargs)
\end{Java}

这两个方法非常重要,第一个方法让反射机制能够访问到私有成员,这使得我们操作对象更加灵活,同时安全性更低。第二个方法可以让我们通过构造器创建对象,相当于直接 \texttt{new} 一个对象。正因为这两个方法,放射机制才如此强大,同时安全性如此低。

同时也有一些常规方法:

\begin{Java}
// 获取对应的 Class 对象
@Override
public Class<T> getDeclaringClass() {
    return clazz;
}
// 获取对应的 Class 对象名
@Override
public String getName() {
    return getDeclaringClass().getName();
}
// 获取参数类型,还有几个类似的,返回值不同
@Override
public Class<?>[] getParameterTypes() {
    return parameterTypes.clone();
}
\end{Java}

\subsection{Field 与成员变量}

利用反射,我们可以对成员的属性进行操作

\begin{Java}
Class clazz = Class.forName("com.xxx.Hero");
Field f = clazz.getDeclaredField("fieldName");
f.set(...);
\end{Java}

和构造器类似,这里通过 getDeclaredField 获得了一个 Field 对象。\texttt{getField} 方法和前面讲的 \texttt{getConstructor} 方法类似,这里不再赘述。

首先来看一下 \texttt{Field} 的声明:

\begin{Java}
public final class Field extends AccessibleObject implements Member
\end{Java}

\texttt{Field} 中的功能性方法如下

\begin{Java}
// 暴力访问,设置为 true,可忽略访问修饰符
public void setAccessible(boolean flag)
// get/set 方法
public Object get(Object obj) throws IllegalArgumentException, IllegalAccessException
public void set(Object obj, Object value) throws IllegalArgumentException, IllegalAccessException
\end{Java}

对于 get/set 方法,有两点要说明的:
\begin{itemize}
    \item \texttt{obj}: 为什么要加一个 Object 参数,因为我们通过 \texttt{getField} 方法获得的 Field 对象仅存储了作为参数名的 name 和类型 type 两个关键属性。我们并没有将这个 Field 绑定到某个对象上,即这个 Field 是一个有名字的对象,仅此而已。
    \item 基本类型:可以看到这里只能设置 Object 类型,基本类型有专门的方法,例如 \texttt{getInt}, \texttt{setFloat}。
\end{itemize}

\texttt{getDeclaredField} 可以获取本类所有的字段,包括 private 的,但是 不能获取继承来的字段。 (注: 这里只能获取到 private 的字段,但并不能访问该 private 字段的值,除非加上 \texttt{setAccessible(true)} )

\subsection{Method 与成员方法}

利用反射,我们可以对成员的方法进行操作

\begin{Java}
Class clazz = Class.forName("com.xxx.Hero");
Method m = clazz.getDeclaredMethod("setName", String.class);  // Hero 有一个 setName 方法
m.invoke();
\end{Java}

首先看一下 \texttt{Method} 的声明:

\begin{Java}
public final class Method extends Executable
\end{Java}

\texttt{Method} 的功能性方法如下:

\begin{Java}
// 暴力访问,设置为 true,可忽略访问修饰符
public void setAccessible(boolean flag)
// 调用函数
@CallerSensitive
@ForceInline
@IntrinsicCandidate
public Object invoke(Object obj, Object... args) throws IllegalAccessException, IllegalArgumentException, InvocationTargetException
\end{Java}

理解起来和前面一样,没啥好嗦的。

\newpage