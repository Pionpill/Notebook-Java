\section{高精度运算}
\subsection{BigInteger 大整数}

\begin{Java}
public class BigInteger extends Number implements Comparable<BigInteger>
\end{Java}

BigInteger 有两个基础属性:

\begin{Java}
final int signum;     // 存储符号位 1:整  0:0 -1:负
final int[] mag;      // 存储数据
\end{Java}

注意这两个属性都是 final 修饰的,这决定了 \texttt{BigInteger} 是不可变对象。

总的来说,BigInteger 的构造函数有两类:
\begin{Java}
// 通过数组构造
public BigInteger(byte[] val, int off, int len)
public BigInteger(int signum, byte[] magnitude, int off, int len)   // 指定符号位
// 通过字符串构造
public BigInteger(String val, int radix)
\end{Java}

同时还提供了一个静态方法:
\begin{Java}
public static BigInteger valueOf(long val)
\end{Java}

首先看一下加法运算,首先通过 signum 进行了优化,如果符号位不同,则转换为减法运算再处理结果:
\begin{Java}
public BigInteger add(BigInteger val) {
    if (val.signum == 0)
        return this;
    if (signum == 0)
        return val;
    if (val.signum == signum)
        return new BigInteger(add(mag, val.mag), signum);

    int cmp = compareMagnitude(val);    // 忽略符号位进行比较
    if (cmp == 0)
        return ZERO;
    int[] resultMag = (cmp > 0 ? subtract(mag, val.mag)
                       : subtract(val.mag, mag));
    resultMag = trustedStripLeadingZeroInts(resultMag);
    return new BigInteger(resultMag, cmp == signum ? 1 : -1);
}
\end{Java}

其次是减法运算,运算逻辑类似:

\begin{Java}
public BigInteger subtract(BigInteger val) {
    if (val.signum == 0)
        return this;
    if (signum == 0)
        return val.negate();
    if (val.signum != signum)
        return new BigInteger(add(mag, val.mag), signum);

    int cmp = compareMagnitude(val);
    if (cmp == 0)
        return ZERO;
    int[] resultMag = (cmp > 0 ? subtract(mag, val.mag)
                       : subtract(val.mag, mag));
    resultMag = trustedStripLeadingZeroInts(resultMag);
    return new BigInteger(resultMag, cmp == signum ? 1 : -1);
}
\end{Java}

乘除法运算,在这部分的算法就比较复杂了,只给出扩展文献连接:
\begin{Java}
public BigInteger multiply(BigInteger val)
public BigInteger divide(BigInteger val)
\end{Java}

\begin{itemize}
    \item 乘法运算: \url{https://zhuanlan.zhihu.com/p/391716853}
\end{itemize}

类似的,还提供了一些数字处理的常用方法,没什么好讲的。

\subsection{BigDecimal}












\newpage