\section{Annotation 注解}

Java注解它提供了一种安全的类似注释的机制,用来将任何的信息或元数据(metadata)与程序元素(类、方法、成员变量等)进行关联。

Java 注解分为三类:
\begin{itemize}
    \item 标准注解: Java 自带的一些功能性注解。
    \item 元注解: 用于定义注解的注解。
    \item 自定注解: 自己定义的注解(略)。
\end{itemize}

\subsection{标准注解}

\subsubsection*{@Override}

功能: 检查该方法是否是重写方法,如果发现其父类,或者是引用的接口中没有该方法时,会报编译错误。

\begin{Java}
@Target(ElementType.METHOD)
@Retention(RetentionPolicy.SOURCE)
public @interface Override {}
\end{Java}

用法: 直接放在要重写的函数上,没有参数。

\subsubsection*{@Deprecated}

功能: 用于标明被修饰的类或类成员、类方法已经废弃、过时,不建议使用。

\begin{Java}
@Documented
@Retention(RetentionPolicy.RUNTIME)
@Target(value={CONSTRUCTOR, FIELD, LOCAL_VARIABLE, METHOD, PACKAGE, MODULE, PARAMETER, TYPE})
public @interface Deprecated {
    String since() default "";
    boolean forRemoval() default false;
}
\end{Java}

参数说明(两个参数都是 Java9 新增的):
\begin{itemize}
    \item since: 指定已注解的 API 元素已被弃用的版本。
    \item forRemoval: 是否在将来的既定版本中会被删除。
\end{itemize}

\begin{Java}
@Deprecated(since = "1.2", forRemoval = true)
\end{Java}

\subsubsection*{@SuppressWarnings}

功能: 用于关闭对类、方法、成员编译时产生的特定警告。

\begin{Java}
@Target({TYPE, FIELD, METHOD, PARAMETER, CONSTRUCTOR, LOCAL_VARIABLE, MODULE})
@Retention(RetentionPolicy.SOURCE)
public @interface SuppressWarnings {
    String[] value();
}
\end{Java}

value 的常见参数如下:
\begin{itemize}
    \item deprecation: 使用了不赞成使用的类或方法
    \item unchecked: 执行了未检查的转换
    \item fallthrough: 当 Switch 程序块直接通往下一种情况而没有 break
    \item path: 在类路径,源文件路径等中有不存在的路径时的警告
    \item serial: 当在可序列化的类上缺少 serialVersionUID 定义时的警告
    \item finally: 任何 finally 子句不能正常完成时的警告
    \item all: 所有的警告
\end{itemize}

\begin{Java}
// @SuppressWarnings("unchecked")                   // 抑制单类型警告
@SuppressWarnings(value={"unchecked", "rawtypes"})  // 抑制多类型警告
public void addItems(String item){  
  @SuppressWarnings("rawtypes")  
   List items = new ArrayList();  
   items.add(item);  
}
\end{Java}

\subsubsection*{@FunctionalInterface}

用于指示被修饰的接口是函数式接口
\begin{Java}
@Documented
@Retention(RetentionPolicy.RUNTIME)
@Target(ElementType.TYPE)
public @interface FunctionalInterface {}
\end{Java}

函数式接口(Functional Interface)就是一个有且仅有一个抽象方法,但是可以有多个非抽象方法的接口。

\subsection{元注解}

\subsubsection*{@Retention}

定义该注解在哪一个级别可用,在源代码中(SOURCE)、类文件中(CLASS)或者运行时(RUNTIME)。

\begin{Java}
@Documented
@Retention(RetentionPolicy.RUNTIME)
@Target(ElementType.ANNOTATION_TYPE)
public @interface Retention {
    RetentionPolicy value();
}

public enum RetentionPolicy {
    SOURCE , CLASS , RUNTIME
}
\end{Java}

value 的常见参数如下:
\begin{itemize}
    \item RetentionPolicy.SOURCE: 编译时被丢弃,只在源文件中出现
    \item RetentionPolicy.CLASS: 编译器将注解记录在类文件中,但不会加载到JVM中,是默认值
    \item RetentionPolicy.RUNTIME: 注解信息会保留在源文件、类文件中,在执行的时也加载到Java的JVM中,因此可以反射性的读取
\end{itemize}

那怎么来选择合适的注解生命周期呢?\footnote{引用自:\url{https://blog.csdn.net/weixin_42403127/article/details/115999679}}
首先要明确生命周期长度 SOURCE < CLASS < RUNTIME ,所以前者能作用的地方后者一定也能作用。一般如果需要 
在运行时去动态获取注解信息,那只能用 RUNTIME 注解;如果要 在编译时进行一些预处理操作,比如生成一些辅助
代码(如 ButterKnife) ,就用 CLASS注解; 如果 只是做一些检查性的操作,比如 @Override 和
 @SuppressWarnings,则 可选用 SOURCE 注解。

\subsubsection*{@Documented}

作用: 生成文档信息的时候保留注解,对类作辅助说明。

\begin{Java}
@Documented
@Retention(RetentionPolicy.RUNTIME)
@Target(ElementType.ANNOTATION_TYPE)
public @interface Documented {}
\end{Java}

如果一个注解@B,被@Documented标注,那么被@B修饰的类,生成文档时,会显示@B。如果@B没有被@Documented标准,最终生成的文档中就不会显示@B。

\subsubsection*{@Target}

作用: 用于描述注解的使用范围。

\begin{Java}
@Documented
@Retention(RetentionPolicy.RUNTIME)
@Target(ElementType.ANNOTATION_TYPE)
public @interface Target {
    ElementType[] value();
}
\end{Java}

ElementType 是一个枚举类型,它定义了被 @Target 修饰的注解可以应用的范围:
\begin{itemize}
    \item ElementType.Type: 类,接口(包括注解),枚举
    \item ElementType.CONSTRUCTOR: 构造函数
    \item ElementType.PARAMETER: 方法的参数
    \item ElementType.FIELD: 字段或属性
    \item ElementType.METHOD: 方法
    \item ElementType.PACKAGE: 包
    \item ElementType.LOCAL\_VARIABLE: 局部变量
    \item ElementType.TYPE\_PARAMETER: 类型变量
    \item ElementType.TYPE\_USE: 任何适用类型的语句中
\end{itemize}

\subsubsection*{@Inherited}

作用: 子类可以继承父类中的该注解, 自动继承注解类型。 如果注解类型声明中存在 @Inherited 元注解,则注解所修饰类的所有子类都将会继承此注解。

\begin{Java}
@Documented
@Retention(RetentionPolicy.RUNTIME)
@Target(ElementType.ANNOTATION_TYPE)
public @interface Inherited {}
\end{Java}

\subsubsection*{@Repeatable}

作用:注解可以重复使用。

\begin{Java}
@Documented
@Retention(RetentionPolicy.RUNTIME)
@Target(ElementType.ANNOTATION_TYPE)
public @interface Repeatable {
    Class<? extends Annotation> value();
}
\end{Java}

\newpage