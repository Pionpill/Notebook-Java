\section{并发集合}

本节默认读者看过 utils 包下非并发集合,掌握Java的基本数据结构。

\subsection{List}

JUC 只提供了一个并发列表,采用了写时复制策略。

\subsubsection{CopyOnWriteArrayList}

并发列表只有一个 CopyOnWriteList,顾名思义,它是一个线程安全的 ArrayList,对其进行的操作都是在底层的一个复制的数组(快照)上进行的,使用了写时复制策略。

\begin{Java}
// Java17
public class CopyOnWriteArrayList<E> implements List<E>, RandomAccess, Cloneable, java.io.Serializable {
    final transient Object lock = new Object();
    private transient volatile Object[] array;
}
\end{Java}

在 Java8 中使用的时 ReentrantLock(一种可重入独占锁) 实现锁机制,而 Java17 中使用的是 synchronized 锁住普通的 Object 对象 lock。这两种机制实现的效果类似,没有很大差别,正如注释所述:

\begin{center}
\textit{We have a mild preference for builtin monitors over ReentrantLock when either will do.}
\end{center}

CopyOnWriteArrayList 在涉及到修改数组的操作时会使用 synchronized(lock) 的方式上锁,这里的修改是指任何非只读操作,包括增加,替换,排序,删除...

我们看一下 add 操作:

\begin{Java}
public boolean add(E e) {
    synchronized (lock) {
        Object[] es = getArray();
        int len = es.length;
        es = Arrays.copyOf(es, len + 1);
        es[len] = e;
        setArray(es);
        return true;
    }
}
\end{Java}

最关键的是第一步操作,每次调用该操作都是尝试获取 lock 锁,如果有其他线程占用了 lock 锁(不仅是 add 操作),线程会被阻塞。

此外,我们会发现,CopyOnWriteArrayList 没有扩容机制,他在 add 时会直接申请一个新的空间,这非常费时间,set 也类似。

\begin{Java}
public E set(int index, E element) {
    synchronized (lock) {
        Object[] es = getArray();
        E oldValue = elementAt(es, index);
        if (oldValue != element) {
            es = es.clone();
            es[index] = element;
        }
        setArray(es);
        return oldValue;
    }
}
\end{Java}

多线程操作可迭代对象存在弱一致性问题,这个问题是指返回迭代器后,其他线程对 list 的增删改对迭代器是不可见的。下面看看 CopyOnWriteArrayList 是如何解决这个问题的:

\begin{Java}
public Iterator<E> iterator() {
    return new COWIterator<E>(getArray(), 0);
}

static final class COWIterator<E> implements ListIterator<E> {
    // 数组快照
    private final Object[] snapshot;
    // 数组下标游标,用于迭代获取元素
    private int cursor;

    COWIterator(Object[] es, int initialCursor) {
        cursor = initialCursor;
        snapshot = es;
    }

    public boolean hasNext() {
        return cursor < snapshot.length;
    }

    @SuppressWarnings("unchecked")
    public E next() {
        if (! hasNext())
            throw new NoSuchElementException();
        return (E) snapshot[cursor++];
    }

    public int nextIndex() {
        return cursor;
    }

    // Not supported. Always throws UnsupportedOperationException.
    public void remove() {
        throw new UnsupportedOperationException();
    }
    public void set(E e) {
        throw new UnsupportedOperationException();
    }
    public void add(E e) {
        throw new UnsupportedOperationException();
    }
}
\end{Java}

获取 CopyOnWriteArrayList 的迭代器对象时,实际上会返回一个 COWIterator 对象,该对象通过 snapshot 保存了当前 list 的内容。注意 snapshot 本质上只是个指针。

如果在遍历期间,其他线程对该 list 进行了修改,那么 snapshot 就是快照了,因为增删改后list里面的数组被新数组替换了,这时候老数组被snapshot引用。这也说明获取迭代器后,使用该迭代器元素时,其他线程对该list进行的增删改不可见,因为它们操作的是两个不同的数组,这就是弱一致性。

这也间接说明,在调用 System 的 copyOf 后,原数组不会被立刻清除,如果没有引用才会被 GC。此外,COWIterator 对象不允许使用修改操作。

总而言之,CopyOnWriteArrayList使用写时复制的策略来保证list的一致性,而获取—修改—写入三步操作并不是原子性的,所以在增删改的过程中都使用了独占锁,来保证在某个时间只有一个线程能对list数组进行修改。

另外CopyOnWriteArrayList提供了弱一致性的迭代器,从而保证在获取迭代器后,其他线程对list的修改是不可见的,迭代器遍历的数组是一个快照。

\fbox{
    \parbox{0.87\textwidth}{
        \begin{notice}
            一般情况下,不要使用 CopyOnWriteArrayList,它只有一个优点: 保证线程安全。但是写时复制策略会照成很大的时间和空间开销,他没有 ArrayList 的扩容机制,迭代器会保存快照,导致旧的数组空间无法被 GC。只有极少情况,需要保存快照且需要保证线程安全时才会用到它。
        \end{notice}
    }
}

\subsection{Set}

\subsubsection{CopyOnWriteSet}

并发集合(Set),JUC 也只提供了一个 CopyOnWriteSet,从名字就知道他是用的写时复制技术,本质上它是用 CopyOnWriteArrayList 实现的\footnote{最常用的非并发集合是 HashSet,采用 HashMap 实现,这何尝不是一种牛头人行为。}。

\begin{Java}
public class CopyOnWriteArraySet<E> extends AbstractSet<E> implements java.io.Serializable {

    private final CopyOnWriteArrayList<E> al;
    public CopyOnWriteArraySet() {
        al = new CopyOnWriteArrayList<E>();
    }

    public boolean add(E e) {
        return al.addIfAbsent(e);
    }

    public boolean contains(Object o) {
        return al.contains(o);
    }

    public boolean remove(Object o) {
        return al.remove(o);
    }
}
\end{Java}

\subsection{Queue}

JUC 提供了很多队列,多到数量是其他容器数量之和。大致可分为两种:
\begin{itemize}
    \item \textbf{ConcurrentLinkedQueue}: 纯 CAS 实现的非阻塞高并发队列。
    \item \textbf{BlockingQueue}: 实现了 BlockingQueue 接口的阻塞队列,使用锁机制实现,并发性相对低。
\end{itemize}

\subsubsection{ConcurrentLinkedQueue}

顾名思义,ConcurrentLinkedQueue 是使用"单向"链表实现的线程安全队列,出队和入队操作使用 CAS 实现线程安全。

\begin{Java}
public class ConcurrentLinkedQueue<E> extends AbstractQueue<E> implements Queue<E>, java.io.Serializable
\end{Java}

ConcurrentLinkedQueue内部的队列使用单向链表方式实现,其中有两个volatile类型的Node节点分别用来存放队列的首、尾节点。

\begin{Java}
transient volatile Node<E> head;
private transient volatile Node<E> tail;
\end{Java}

默认情况下,head 和 tail 时指向空节点即哨兵节点:

\begin{Java}
public ConcurrentLinkedQueue() {
    head = tail = new Node<E>();
}
\end{Java}

Node 节点很简单,只有两个元素用于存储值和下一个 Node 的指针:

\begin{Java}
volatile E item;
volatile Node<E> next;
\end{Java}

ConcurrentLinkedQueue 有一个特殊之处,tail 节点并不一定是尾节点,也有可能是尾节点的前一个节点。为什么要这样做呢?

\begin{itemize}
    \item 假设我们不这样做,让 tail 节点始终是尾部节点,那么我们每次添加元素都需要对 tail 进行 CAS 操作(注意只是 tail 节点)。这样的好处是代码简单,缺陷是频繁的 CAS 操作效率低。
    \item 如果我们每两个或者多个节点对 tail 节点进行 CAS 操作,就可以减少 CAS 更新 tail 节点的次数,提高入队的效率。
\end{itemize}

在 Java1.7 中,可以使用 hops 变量控制 tail 节点每添加多少新节点才更新,这是一种优化策略。hops 值小则需要频繁的对 tail 进行 CAS 操作,过大则需要通过链表遍历查找尾节点,violate 修饰的 tail 节点对读有更好的优化。因此在 Java8 之后,默认添加两个节点才更新 tail 节点。

当然,这样做提高性能的同时也是有代价的: 极大地增加了代码复杂度,不借助资料看懂基本看不懂,借助资料看源码也很迷。

此外,由于采用 CAS 操作没有加锁,在计算队列长度时有可能出错,因为计算过程中,有可能增加或者删除元素节点。

到此,从应用层面我们只需要记住以下几点:
\begin{itemize}
    \item ConcurrentLinkedQueue 使用单向链表存储数据。
    \item ConcurrentLinkedQueue 采用 CAS 保证原子性,因此可能会导致 size 计算失误。
    \item ConcurrentLinkedQueue 对 tail 节点进行了优化,每添加两次元素,tail 节点移动一次,减少 CAS 操作。
\end{itemize}

\subsubsection{LinkedBlockingQueue}

从名字上可以知道,LinkedBlockingQueue 是链表实现的阻塞队列。既然是阻塞,那肯定是通过上锁保证原子性。

\begin{Java}
public class LinkedBlockingQueue<E> extends AbstractQueue<E> implements BlockingQueue<E>, java.io.Serializable
\end{Java}

BlockingQueue 定义了基础的 queue 方法,功能上它有点像标记接口,定义了一套阻塞方法。

LinkedBlockingQueue 有以下几个关键属性:

\begin{Java}
private final AtomicInteger count = new AtomicInteger();
transient Node<E> head;
private transient Node<E> last;
private final ReentrantLock takeLock = new ReentrantLock();
private final Condition notEmpty = takeLock.newCondition();
private final ReentrantLock putLock = new ReentrantLock();
private final Condition notFull = takeLock.newCondition();

public LinkedBlockingQueue(int capacity) {
    if (capacity <= 0) throw new IllegalArgumentException();
    this.capacity = capacity;
    last = head = new Node<E>(null);
}
\end{Java}

和 ConcurrentLinkedQueue 一样保存了头节点和尾节点,同时维持一个 count 属性用于计算节点数量。此外为了线程安全创造了两个 ReentrantLock 实例,分别用来控制元素入队和出队的原子性:
\begin{itemize}
    \item \textbf{takeLock}: 控制同时只有一个线程可以从队列头获取元素,其他线程必须等待。
    \item \textbf{putLock}: 控制同时只能有一个线程可以获取锁,在队列尾部添加元素,其他线程必须等待。
\end{itemize}

notEmpty和notFull是条件变量,它们内部都有一个条件队列用来存放进队和出队时被阻塞的线程。

由于条件变量notEmpty内部的条件队列的维护使用的是takeLock的锁状态管理机制,所以在调用notEmpty的await和signal方法前调用线程必须先获取到takeLock锁,否则会抛出IllegalMonitorStateException异常。notEmpty内部则维护着一个条件队列,当线程获取到takeLock锁后调用notEmpty的await方法时,调用线程会被阻塞,然后该线程会被放到notEmpty内部的条件队列进行等待,直到有线程调用了notEmpty的signal方法。noFull 类似。

LinkedBlockingQueue 的原理非常简单,就是采用 ReentrantLock,这里只贴出 offer() 的部分代码:

\begin{Java}
public boolean offer(E e) {
    // 省略一些检查,初始化操作
    putLock.lock();
    try {
        if (count.get() == capacity)
            return false;
        enqueue(node);
        c = count.getAndIncrement();
        if (c + 1 < capacity)
            notFull.signal();
    } finally {
        putLock.unlock();
    }
    // ...
    return true;
}
\end{Java}

\begin{figure}[H]
    \centering
    \small
    \begin{tikzpicture}[scale = 1]
        \node[color = white, fill = green!60!black] (head) at (0,0) {head};
        \node[color = white, fill = brown!80] (takeLock) at (0,-1.5) {takeLock};
        \draw [-Stealth] (head) -- (takeLock);
        \node[color = white, fill = green!60!black] (node1) at (2,0) {node1};
        \node[color = white, fill = green!60!black] (node2) at (4,0) {node2};
        \node[color = white, fill = green!60!black] (tail) at (6,0) {tail};
        \node[color = white, fill = brown!80] (putLock) at (6,-1.5) {putLock};
        \draw [-Stealth] (tail) -- (putLock);
        \draw [-Stealth] (head) -- (node1);
        \draw [-Stealth](node1) -- (node2);
        \draw [dashed] (node2) -- (tail);
        \node (peek) at (-1.5,1) {peek()};
        \node (take) at (-2,0) {take()};
        \node (poll) at (-1.5,-1) {poll()};
        \foreach \node in {peek, take, poll} {
            \draw [-Stealth] (\node) -- (head);
        }
        \node (offer) at (7.5,1) {offer()};
        \node (put) at (7.5,-1) {put()};
        \foreach \node in {offer, put} {
            \draw [-Stealth] (\node) -- (tail);
        }
        \begin{scope}[font=\scriptsize]
            \node[color = white, fill = brown!60] (takeLock AQS) at (-1,-3) {AQS 阻塞队列};
            \node[color = white, fill = brown!60, text width=4em] (notEmpty 条件队列) at (1,-3) {notEmpty 条件队列};
            \node[color = white, fill = brown!60] (putLock AQS) at (5,-3) {AQS 阻塞队列};
            \node[color = white, fill = brown!60, text width=4em] (notFull 条件队列) at (7,-3) {notFull 条件队列};
        \end{scope}
        \draw [dotted, {Stealth[round]}-{Stealth[round]}, align = center] (takeLock) -- (takeLock AQS);
        \draw [dotted, {Stealth[round]}-{Stealth[round]}, align = center] (takeLock) -- (notEmpty 条件队列);
        \draw [dotted, {Stealth[round]}-{Stealth[round]}, align = center] (putLock) -- (putLock AQS);
        \draw [dotted, {Stealth[round]}-{Stealth[round]}, align = center] (putLock) -- (notFull 条件队列);
    \end{tikzpicture}
    \caption{BlockingLinkedQueue}
    \label{fig:BlockingLinkedQueue}
\end{figure}

\subsubsection{ArrayBlockingQueue}

ArrayBlockingQueue 和 LinkedBlockingQueue 类似,只不过底层采用数组存储元素,这节只讲不同之处。

\begin{Java}
public class ArrayBlockingQueue<E> extends AbstractQueue<E> implements BlockingQueue<E>, java.io.Serializable {
    final Object[] items;
    int takeIndex;
    int putIndex;
    int count;
    final ReentrantLock lock;
    @SuppressWarnings("serial")
    private final Condition notEmpty;
    @SuppressWarnings("serial")
    private final Condition notFull;
}
\end{Java}

由于采用数组存储元素,ArrayBlockingQueue 是有界的,因此需要在构造函数传入队列大小参数:

\begin{Java}
public ArrayBlockingQueue(int capacity, boolean fair) {
    if (capacity <= 0)
        throw new IllegalArgumentException();
    this.items = new Object[capacity];  // items 是存储元素的数组
    lock = new ReentrantLock(fair);     // 默认情况下,采用非公平策略
    notEmpty = lock.newCondition();
    notFull =  lock.newCondition();
}
\end{Java}

和写时复制以及扩容机制不同,ArrayBlockingQueue 数组的大小是不会变的,如果超出数组大小会出现错误:

\begin{Java}
public boolean offer(E e) {
    Objects.requireNonNull(e);
    final ReentrantLock lock = this.lock;
    lock.lock();
    try {
        if (count == items.length)
            return false;
        else {
            enqueue(e);
            return true;
        }
    } finally {
        lock.unlock();
    }
}
\end{Java}

保证线程安全的策略和 LinkedBlockingQueue 类似,采用锁策略,不过 ArrayBlockingQueue 只有一个锁:

\begin{Java}
final ReentrantLock lock;
\end{Java}

所以操作,包括读写,转换都会用到 lock 锁,因此效率是比较低的,由于采用了锁机制,tail,head 都没有被 violate 修饰,因为锁可以保证内存可见性。

总而言之,在保证线程安全方面,ArrayBlockingQueue 采用了简单的直接对所有方法加锁(ReentrantReadWriteLock)的方式。不同于 LinkedBlockingQueue 对头部和尾部采用了不同的锁,ArrayBlockingQueue 只采用了一个锁(两个条件变量),效率明显不及 LinkedBlockingQueue。但由于 ArrayBlockingQueue 采用定长数组存储数据,可以实现随机查找,这方面的优势较大。

\begin{figure}[H]
    \centering
    \small
    \begin{tikzpicture}[scale = 1]
        \node[color = white, fill = green!60!black] (item1) at (0,0) {item1};
        \node[color = white, fill = green!60!black] (item2) at (1.2,0) {item2};
        \node[color = white, fill = green!60!black] (item3) at (2.4,0) {item3};
        \node[color = white, fill = green!60!black] (item4) at (3.6,0) {item4};
        \node[color = white, fill = green!60!black] (itemN) at (6,0) {itemN};
        \draw [dashed] (item4) -- (itemN);
        \node (peek) at (-1.5,1) {peek()};
        \node (take) at (-2,0) {take()};
        \node (poll) at (-1.5,-1) {poll()};
        \foreach \node in {peek, take, poll} {
            \draw [-Stealth] (\node) -- (item1);
        }
        \node (offer) at (7.5,1) {offer()};
        \node (put) at (7.5,-1) {put()};
        \foreach \node in {offer, put} {
            \draw [-Stealth] (\node) -- (itemN);
        }
        \draw[decorate,decoration={brace,mirror,raise=10pt,amplitude=2mm}] (0,0) -- (6,0);
        \node[color = white, fill = brown!80] (lock) at (3,-1) {lock};
        \node[color = white, fill = brown!60, text width=3em, align=center, font = \scriptsize] (AQS) at (1,-2.5) {AQS 阻塞队列};
        \node[color = white, fill = brown!60, text width=3em, align=center, font = \scriptsize] (notEmpty) at (3,-2.5) {notEmpty 条件队列};
        \node[color = white, fill = brown!60, text width=3em, align=center, font = \scriptsize] (notFull) at (5,-2.5) {notFull 条件队列};\
        \draw [dotted, {Stealth[round]}-{Stealth[round]}, align = center] (lock) -- (notFull);
        \draw [dotted, {Stealth[round]}-{Stealth[round]}, align = center] (lock) -- (notEmpty);
        \draw [dotted, {Stealth[round]}-{Stealth[round]}, align = center] (lock) -- (AQS);
    \end{tikzpicture}
    \caption{BlockingArrayQueue}
    \label{fig:BlockingArrayQueue}
\end{figure}


\subsubsection{PriorityBlockingQueue}

PriorityBlockingQueue 是带优先级的无界阻塞队列,其内部使用平衡二叉树实现,直接遍历元素不保证有序。默认使用对象的 compareTo 方法提供比较规则。扩容机制参考 ArrayList,优先机制参考 TreeMap,阻塞机制参考 ArrayBlockingQueue。这节只讲多线程下的不同点。

\begin{Java}
public class PriorityBlockingQueue<E> extends AbstractQueue<E> implements BlockingQueue<E>, java.io.Serializable {
    private transient Object[] queue;
    private transient int size;
    private transient Comparator<? super E> comparator;
    private final ReentrantLock lock = new ReentrantLock();
    @SuppressWarnings("serial")
    private final Condition notEmpty = lock.newCondition();
    private transient volatile int allocationSpinLock;
}
\end{Java}

可以发现,PriorityBlockingQueue 种有一个 queue 数组用于存放元素,由于是无界队列,queue 是可以自动扩容的。allocationSpinLock 是一个自旋锁,通过 CAS 操作保证同时只有一个线程可以扩容队列,状态为 0 或 1,0 表示没有进行扩容,1表示正在进行扩容。

\begin{Java}
public PriorityBlockingQueue(int initialCapacity, Comparator<? super E> comparator) {
    if (initialCapacity < 1)
        throw new IllegalArgumentException();
    this.comparator = comparator;
    this.queue = new Object[Math.max(1, initialCapacity)];  // 默认初始大小为 11
}
\end{Java}

在扩容过程中,PriorityBlockingQueue 会检查数组大小,如果小于 64,容量+2, 大于 64 容量翻倍,扩容代码如下:

\begin{Java}
private void tryGrow(Object[] array, int oldCap) {
    lock.unlock(); // must release and then re-acquire main lock
    Object[] newArray = null;
    if (allocationSpinLock == 0 && ALLOCATIONSPINLOCK.compareAndSet(this, 0, 1)) {
        try {
            int growth = (oldCap < 64) ? (oldCap + 2) : (oldCap >> 1);
            int newCap = ArraysSupport.newLength(oldCap, 1, growth);
            if (queue == array)
                newArray = new Object[newCap];
        } finally {
            allocationSpinLock = 0;
        }
    }
    if (newArray == null) // back off if another thread is allocating
        Thread.yield();
    lock.lock();
    if (newArray != null && queue == array) {
        queue = newArray;
        System.arraycopy(array, 0, newArray, 0, oldCap);
    }
}
\end{Java}

我们发现在进入扩容是首先会释放锁,然后使用 CAS 控制只有一个扩容可以成功。为什么要释放锁呢? 这是一个优化方案,不释放锁对功能没有影响,但是扩容是一个耗时的过程,如果不释放锁,在扩容期间就不能进行出队和入队操作,这会大大降低并发性。


此外,虽然 PriorityBlockingQueue 和 ArrayList 一样没有动态缩容机制,但是 ArrayList 提供了公有的 trimToSize() 方法 用于手动缩容,而 PriorityBlockingQueue 没有提供任何缩容方法。

具体的二叉树算法这里不介绍。

\begin{figure}[H]
    \centering
    \small
    \begin{tikzpicture}[scale = 1]
        \node[color = white, fill = green!60!black] (item1) at (0,0) {item1};
        \node[color = white, fill = green!60!black] (item2) at (1.2,0) {item2};
        \node[color = white, fill = green!60!black] (item3) at (2.4,0) {item3};
        \node[color = white, fill = green!60!black] (item4) at (3.6,0) {item4};
        \node[color = white, fill = green!60!black] (itemN) at (6,0) {itemN};
        \draw [dashed] (item4) -- (itemN);
        \node (peek) at (-1.5,1) {peek()};
        \node (take) at (-2,0) {take()};
        \node (poll) at (-1.5,-1) {poll()};
        \foreach \node in {peek, take, poll} {
            \draw [-Stealth] (\node) -- (item1);
        }
        \node (offer) at (7.5,1) {offer()};
        \node (put) at (7.5,-1) {put()};
        \foreach \node in {offer, put} {
            \draw [-Stealth] (\node) -- (itemN);
        }
        \draw[decorate,decoration={brace,mirror,raise=10pt,amplitude=2mm}] (0,0) -- (6,0);
        \node[color = white, fill = brown!80] (lock) at (3,-1) {lock};
        \node[color = white, fill = brown!60, text width=3em, align=center, font = \scriptsize] (AQS) at (2,-2.5) {AQS 阻塞队列};
        \node[color = white, fill = brown!60, text width=3em, align=center, font = \scriptsize] (notEmpty) at (4,-2.5) {notEmpty 条件队列};
        \draw [dotted, {Stealth[round]}-{Stealth[round]}, align = center] (lock) -- (notEmpty);
        \draw [dotted, {Stealth[round]}-{Stealth[round]}, align = center] (lock) -- (AQS);
    \end{tikzpicture}
    \caption{PriorityBlockingQueue}
    \label{fig:PriorityBlockingQueue}
\end{figure}

\subsubsection{DelayQueue}

DelayQueue并发队列是一个无界阻塞延迟队列,队列中的每个元素都有个过期时间,当从队列获取元素时,只有过期元素才会出队列。队列头元素是最快要过期的元素。

\begin{Java}
public class DelayQueue<E extends Delayed> extends AbstractQueue<E> implements BlockingQueue<E> {
    private final transient ReentrantLock lock = new ReentrantLock();
    private final PriorityQueue<E> q = new PriorityQueue<E>();
    private Thread leader;
    private final Condition available = lock.newCondition();
    public DelayQueue() {}
}
\end{Java}

DelayQueue 内部使用 PriorityQueue 存放数据,使用 ReentrantLock 实现线程同步。加入 DelayQueue 的元素都需要实现 Delayed 接口,由于每个元素都有一个过期时间,所以要实现获知当前元素还剩下多少时间就过期了的接口,由于内部使用优先级队列来实现,所以要实现元素之间相互比较的接口。:

\begin{Java}
public interface Delayed extends Comparable<Delayed> {
    long getDelay(TimeUnit unit);
}
\end{Java}

具体方法上,没什么好说的,就是用 ReentrantLock 加锁然后调用 PriorityQueue 的方法。poll 和 take 方法比较特殊,会查看队列元素是否过期,过期了才取。

\begin{figure}[H]
    \centering
    \small
    \begin{tikzpicture}[scale = 1]
        \draw[color = black!40,draw=white,fill = green!10] (-0.65,0.7) rectangle (6.65,-0.32);
        \node[font=\footnotesize] at (3,0.5) {PriorityQueue};
        \node[color = white, fill = green!60!black] (item1) at (0,0) {item1};
        \node[color = white, fill = green!60!black] (item2) at (1.2,0) {item2};
        \node[color = white, fill = green!60!black] (item3) at (2.4,0) {item3};
        \node[color = white, fill = green!60!black] (item4) at (3.6,0) {item4};
        \node[color = white, fill = green!60!black] (itemN) at (6,0) {itemN};
        \draw [dashed] (item4) -- (itemN);
        \node (peek) at (-1.5,1) {peek()};
        \node (take) at (-2,0) {take()};
        \node (poll) at (-1.5,-1) {poll()};
        \foreach \node in {peek, take, poll} {
            \draw [-Stealth] (\node) -- (item1);
        }
        \node (offer) at (7.5,1) {offer()};
        \node (put) at (7.5,-1) {put()};
        \foreach \node in {offer, put} {
            \draw [-Stealth] (\node) -- (itemN);
        }
        \draw[decorate,decoration={brace,mirror,raise=10pt,amplitude=2mm}] (0,0) -- (6,0);
        \node[color = white, fill = brown!80] (lock) at (3,-1) {lock};
        \node[color = white, fill = brown!60, text width=3em, align=center, font = \scriptsize] (AQS) at (2,-2.5) {AQS 阻塞队列};
        \node[color = white, fill = brown!60, text width=3em, align=center, font = \scriptsize] (available) at (4,-2.5) {available 条件队列};
        \draw [dotted, {Stealth[round]}-{Stealth[round]}, align = center] (lock) -- (available);
        \draw [dotted, {Stealth[round]}-{Stealth[round]}, align = center] (lock) -- (AQS);
    \end{tikzpicture}
    \caption{DelayQueue}
    \label{fig:DelayQueue}
\end{figure}

Queue 到这里就介绍完了,对入门来说非常详细,并发容器中还有几个 Deque 其原理和对应的 Queue 类似,不再赘述。 

\subsection{Map}

\subsubsection{ConcurrentHashMap}

本文默认讲 Java8 的 ConcurrentHashMap 逻辑,具体代码为 Java17。ConcurrentHashMap 的实现非常复杂,源码有 6k+ 行,因此源码解析和 Java7 的实现方案不再赘述(写不下)。

ConcurrentHashMap 基于 HashMap 逻辑,实现了针对数组元素的细粒度锁,相比 HashTable 有更好的并发性。

\begin{Java}
public class ConcurrentHashMap<K,V> extends AbstractMap<K,V> implements ConcurrentMap<K,V>, Serializable
\end{Java}

\paragraph*{锁粒度}

我们知道,HashTable 使用 synchronized 锁对象的方式解决了多线程问题,但如果以整个容器为一个资源进行锁定,那么就变为了串行操作。而根据hash表的特性,具有冲突的操作只会出现在同一槽位,而与其它槽位的操作互不影响。

因此可以将资源锁粒度缩小到槽位(数组节点)上,这样热点一分散,冲突的概率就大大降低,并发性能就能得到很好的增强。

ConcurrentHashMap使用JUC包中通过直接操作内存中的对象,将比较与替换合并为一个原子操作的乐观锁形式(CAS)来进行简单的值替换操作,对于一些含有复杂逻辑的流程对Node节点对象使用synchronize进行同步。

\paragraph*{并发扩容}

ConcurrentHashMap 扩容采用多线程扩容方式提高了扩容的效率,扩容时不允许进行其他操作。在扩容过程中,ConcurrentHashMap 会被分成多端由不同的空闲线程将各段存储到新的容器中。

\paragraph*{主要方法} put get 方法的逻辑与主要代码

在执行 put 方法时,主要使用如下逻辑:
\begin{itemize}
    \item 由 spread 函数根据 key 获取 hash 值。
    \item 判断数组是否需要初始化。
    \item 定位到数组的 Node 节点 f。
    \begin{itemize}
        \item 槽位为空(f 为 null),采用 CAS 方式添加。
        \item 如果处于扩容阶段,线程协助扩容。
        \item 其他情况,通过 synchronized 锁住节点,根据结构不同(链表或红黑树)插入新界节点。
    \end{itemize}
    \item 链表长度达到 8 ,进行扩容。
\end{itemize}

\begin{Java}
final V putVal(K key, V value, boolean onlyIfAbsent) {
    if (key == null || value == null) throw new NullPointerException();
    // 计算 hash 值
    int hash = spread(key.hashCode());
    for (Node<K,V>[] tab = table;;) {
        Node<K,V> f; int n, i, fh; K fk; V fv;
        // 判断是否要初始化
        if (tab == null || (n = tab.length) == 0)
            tab = initTable();
        // 空槽位,添加节点
        else if ((f = tabAt(tab, i = (n - 1) & hash)) == null) {
            if (casTabAt(tab, i, null, new Node<K,V>(hash, key, value)))
                break;                   
        }
        // 处于扩容阶段
        else if ((fh = f.hash) == MOVED)
            tab = helpTransfer(tab, f);
        // 非空槽位,锁住对应槽位并添加新节点
        else {
            synchronized (f) {
                // 省略链表与红黑树添加节点具体操作
            }
            if (binCount != 0) {
                // 链表长度大于8,转换为红黑树
                if (binCount >= TREEIFY_THRESHOLD)
                    treeifyBin(tab, i);
                if (oldVal != null)
                    return oldVal;
                break;
            }
        }
    }
    addCount(1L, binCount);
    return null;
}
\end{Java}

get 和 hashMap 类似,没有加锁,因为 Node 元素是 volatile 修饰的:

\begin{Java}
public V get(Object key) {
    Node<K,V>[] tab; Node<K,V> e, p; int n, eh; K ek;
    // 计算 hash 值
    int h = spread(key.hashCode());
    // 数组是否为空,槽位是否为空
    if ((tab = table) != null && (n = tab.length) > 0 &&
        (e = tabAt(tab, (n - 1) & h)) != null) {
        // 在数组上,直接放回
        if ((eh = e.hash) == h) {
            if ((ek = e.key) == key || (ek != null && key.equals(ek)))
                return e.val;
        }
        // 红黑树结构
        else if (eh < 0)
            return (p = e.find(h, key)) != null ? p.val : null;
        // 链表结构
        while ((e = e.next) != null) {
            if (e.hash == h &&
                ((ek = e.key) == key || (ek != null && key.equals(ek))))
                return e.val;
        }
    }
    return null;
}
\end{Java}



看样子已经完成被 CopyOnWriteArrayList 填满了。

\subsection*{并发集合总结}

\begin{table}[H]
    \centering
    \small
    \caption{并发集合总结}
    \label{table:并发集合总结}
    \setlength{\tabcolsep}{4mm}
    \begin{tabular}{l|l|l|llllll}
        \toprule
        \textbf{数据结构} & \textbf{类/获取方式} & \textbf{锁类型} & \textbf{技术} & \textbf{适用情形} \\ 
        \midrule
        \multirow{2}{*}{List} & Collections.synchronizedList(List) & synchronized & 锁对象 & - \\
         & CopyOnWriteriteArrayList & synchronized & 写时复制 & 读多写少 \\
        \midrule
        \multirow{2}{*}{Set} & Collections.synchronizedMap(Map) & synchronized & 锁对象 & - \\
         & CopyOnWriteSet & synchronized & 写时复制 & 读多写少 \\
        \midrule
        \multirow{2}{*}{Queue} & ConcurrentLinkedQueue & CAS & - & 默认选项 \\
         & BlockingQueue & ReentrantLock & 数组或元素锁 & 读多写少 \\
         \midrule
        \multirow{2}{*}{Map} & ConcurrentHashMap & CAS+syn & 锁槽位 & 默认选项 \\
         & HashTable & synchronized & 锁对象 & - \\
          
        \bottomrule
    \end{tabular}
\end{table}

针对 Queue 和 Map 很好选择: ConcurrentLinkedQueue 和 ConcurrentHashMap,极少情况才会考虑其他选项。比较特殊的 Stack,非并发情况下使用 LinkedList,否则使用 Stack。而 List 和 Set 则需要慎重考虑:

针对 List,由于 CopyOnWriteArrayList 采用的是写时复制策略,在数据量大或者需要频繁修改数据的情况,性能非常低。因此我们首先要考虑的是也不应该采用高并发 List,能否采用其他数据结构或者将 List 放入非并发情况下。如果一定要用,则需要知道写时复制的缺陷,如果不能接受,建议派生 ArrayList 或用 Collections.synchronizedList 构建并发列表,最后才考虑 CopyOnWriteArrayList。

针对 Set,同样需要考虑写时复制的缺陷。Java 没有给出高效率的并发 List 和 Set 是有原因的,这两个数据结构在多线程中并没有其他两个常用,因此在设计之初就需要考虑是否真的有必要使用这两种数据结构的并发版本。

\newpage