\section{锁,可见性与原子操作}

\subsection{并发概念}

\subsubsection*{并发,并行与多线程安全}

首先区分一下并发与并行的关系:
\begin{itemize}
    \item \textbf{并发}: 同一时间段内多个任务同时执行,并不是同一时间点同时进行。
    \item \textbf{并行}: 单位时间内多个任务同时执行。
\end{itemize}

并行与并发的概念源于单 CPU 时代,再单个 CPU(不考虑超线程技术) 上处理多个线程,多个线程之间轮流获取 CPU 资源,此时可以说并发执行。而在多 CPU 时代,多个 CPU 同一时间点处理多个线程,这可以说是并行处理。再实际编程中,线程数是多余 CPU 个数,所以一般称为多线程并发编程。

并发编程会带来线程安全问题,其本质是对共享线程的非原子性操作。例如我们有这样一个操作(例子,实际情况并不一定如此): 让某个对象自增,这个操作在 CPU 中会被分成三个步骤: 
\begin{itemize}
    \item 1.从内存获取对象,
    \item 2.在 CPU 中让对象自增,
    \item 3.将对象写回内存。
\end{itemize}

多线程下会发生什么事情呢,假设有两个线程A,B 都需要执行以上三个步骤,执行顺序如下: A1 $\rightarrow$ A2 $\rightarrow$ B1 $\rightarrow$ A3 $\rightarrow$ B2 $\rightarrow$ B3。 由于B线程取对象时,A线程尚未将对象写回内存,因此最后该对象在内存中只被自增了一次(虽然 CPU 执行了两次自增)。

大部分未经处理的的 Java 操作都是非源自性的,count++ 就不是源自操作。为了解决这个问题,Java 提供了各种数据同步方案,会在下文一一介绍。

\subsubsection*{Java 内存模型}

接下来了解一下 Java 内存模型,Java内存模型规定,将所有的变量都存放在主内存中,当线程使用变量时,会把主内存里面的变量复制到自己的工作空间或者叫作工作内存,线程读写变量时操作的是自己工作内存中的变量。

更具体的,可以认为主内存在物理机的内存条,而线程的工作内存是 L1 缓存,L2 缓存是所有 CPU 共享的。L1缓存,L2缓存,寄存器都可以是工作内存。当一个线程操作共享变量时,它首先从主内存复制共享变量到自己的工作内存,然后对工作内存里的变量进行处理,处理完后将变量值更新到主内存。线程安全问题就发生在工作内存与主内存交互数据的过程中。

另外,CPU 进行上下文切换时也会消耗资源,因为需要将工作内存与主内存中数据交换。如果是单 CPU,那么多线程并不能带来效率提升,反而会降低效率。

内存可见性是指当一个线程修改了某个变量的值,其它线程总是能知道这个变量变化。

\subsection{synchronized 与 volatile 关键字}

\subsubsection*{synchronized 关键字}

synchronized 块是 Java 提供的一种原子性内置锁, Java中的每个对象都可以把它当作一个同步锁来使用。这些Java内置的使用者看不到的锁被称为内部锁,也叫作监视器锁。

线程的执行代码在进入synchronized代码块前会自动获取内部锁,这时候其他线程访问该同步代码块时会被阻塞。拿到内部锁的线程会在正常退出同步代码块或者抛出异常后或者在同步块内调用了该内置锁资源的 wait 系列方法时释放该内置锁。

内置锁是排它锁,也就是当一个线程获取这个锁后,其他线程必须等待该线程释放锁后才能获取该锁。

Java中的线程是与操作系统的原生线程一一对应的,所以当阻塞一个线程时,需要从用户态切换到内核态\footnote{操作系统针对硬件与用户程序的两种操作状态.}执行阻塞操作,这是很耗时的操作,而synchronized的使用就会导致上下文切换。

synchronized 的内存语义可以将 synchronized 块内使用到的变量从线程的工作内存中清除. 这样在synchronized块内使用到该变量时就不会从线程的工作内存中获取,而是直接从主内存中获取。退出synchronized块的内存语义是把在synchronized块内对共享变量的修改刷新到主内存。

说人话一点, 使用到了 synchronized 的资源, 线程操作该资源有如下变动:
\begin{itemize}
    \item 线程获取锁: 从主存中获取资源.
    \item 线程释放锁: 将资源刷新到主存, \textbf{清空工作内存中对应资源数据}.
\end{itemize}

这也是加锁和释放锁的语义,当获取锁后会清空锁块内本地内存中将会被用到的共享变量,在使用这些共享变量时从主内存进行加载,在释放锁时将本地内存中修改的共享变量刷新到主内存。

除可以解决共享变量内存可见性问题外,synchronized经常被用来实现原子性操作。另外请注意,synchronized关键字会引起线程上下文切换并带来线程调度开销。

synchronized 可以修饰四种不同类型的代码块:
\begin{itemize}
    \item 实例方法: 锁对象是 this,即对象本身。
    \item 静态方法:锁对象是类,JVM 中只有一个类,一个类只能有一个线程在运行。
    \item 代码块: 锁对象是 () 中的对象。
    \item 静态代码块: 锁对象是类。
\end{itemize}

\subsubsection*{volatile 关键字}

使用 synchronized 锁方式可以解决共享内存可见性问题,但是锁太笨重,切换上下文会带来很大开销. 对于解决内存可见性问题,Java还提供了一种弱形式的同步,也就是使用volatile关键字。该关键字可以确保对一个变量的更新对其他线程马上可见,同时不会加锁。

当一个变量被声明为volatile时,线程在写入变量时不会把值缓存在寄存器或者其他地方,而是会把值刷新回主内存。当其他线程读取该共享变量时,会从主内存重新获取最新值,而不是使用当前线程的工作内存中的值。可以理解为在写入变量时\textbf{不再使用CPU缓存},注意计算时还是要用。

\begin{Java}
public class ThreadSafeInteger {
    private volatile int value;
    public int get() {
        return value;
    }        
    public void set(int value) {
        this.value = value;
    }
}
\end{Java}

相比 synchronized 关键字, volatile 有如下优缺点:
\begin{itemize}
    \item 不需要加锁,不会阻塞其他线程。
    \item 不能保证操作的原子性。
\end{itemize}

那么什么情况下可以不用 synchronized 而用 volatile 关键字呢: 写入变量不依赖变量当前值,即不存在计算过程。

此外 volatile 只能修饰成员变量。

\subsection{原子性,CAS操作与Unsafe类}

原子性操作是指,一系列操作,要么全部执行,要么全部不执行。比如前面说过的 count++ 实质上有三个步骤,如果我们使用 java -c 查看汇编代码,会发现有四个步骤,所以 count++ 不是一个原子操作。

保证原子性最简单(目前只知道)的方法是使用 synchronized 关键字,但是 synchronized 加锁会有较高的性能开销,而且会造成线程阻塞。如果使用 volatile 关键字可以保证部分内存可见性,但是不能解决``读-改-写''原子性问题。

\subsubsection*{CAS 操作}

CAS(Compare and Swap) 是 JDK 提供的非阻塞原子性操作,通过\textbf{硬件}保证了比较-更新操作的原子性。JDK里面的Unsafe类提供了一系列的compareAndSwap*方法,下面看下 compareAndSwapLong 方法(了解,许多原子操作通过 CAS 实现,一般不会手动调用 CAS 操作):

\begin{itemize}
\item boolean compareAndSwapLong(Object obj,long valueOffset,long expect, long update)
    \begin{itemize}
        \item obj: 对象内存位置。
        \item valueOffset: 对象中的变量的偏移量。
        \item expect: 变量预期值。
        \item update: 新的值。
    \end{itemize}
\end{itemize}

其操作含义是,如果对象obj中内存偏移量为valueOffset的变量值为expect,则使用新的值update替换旧的值expect。这是处理器提供的一个原子性指令。

CAS 存在 ABA 问题,简单阐述如下: 线程 I 将数据 A 变成了 B,然后又重新变成了 A,此时另一个线程读取该数据时,发现 A 没有变化,就误认为是原来的 A,到那时 A 的一些属性或者状态已经发生了变化。

ABA问题的产生是因为变量的状态值产生了环形转换,就是变量的值可以从A到B,然后再从B到A。如果变量的值只能朝着一个方向转换,比如A到B,B到C,不构成环形,就不会存在问题。JDK中的AtomicStampedReference类给每个变量的状态值都配备了一个时间戳,从而避免了ABA问题的产生。

\subsubsection*{Unsafe 类}

Unsafe 提供了很多硬件级别的原子性操作,类中的方法本质上都是 native 方法,通过 JNI 访问本地 C++ 实现库,几个需要了解的方法有:
\begin{itemize}
    \item long objectFieldOffset(Field f)
    
访问指定的变量在所属类中的内存偏移地址,该地址仅仅在该 Unsafe 函数中使用。
    \item int arrayBaseOffset(Class<?> arrayClass)
    
获取第一个数组元素的地址。
    \item int arrayIndexScale(Class<?> arrayClass)
    
获取数组中一个元素占用的字节。
    \item long getLongVolatile(Object o, long offset)
    
获取对象obj中偏移量为offset的变量对应volatile语义的值。
    \item void putLongVolatile(Object o, long offset, long x)
    
设置obj对象中offset偏移的类型为long的field的值为value,支持volatile语义。
    \item void putOrderedLong(Object o, long offset, long x)
    
有延迟的putLongvolatile方法,并且不保证值修改对其他线程立刻可见。只有在变量使用volatile修饰并且预计会被意外修改时才使用该方法。
    \item void park(boolean isAbsolute, long time)
    
阻塞当前线程,time 表示经过多长时间被唤醒。isAbsolute 用于和 time 配合产生不同效果。
    \item void unpark(Object thread)
    
唤醒被 park 阻塞的线程。
    \item long getAndSetLong(Object o, long offset, long newValue)
    
获取对象obj中偏移量为offset的变量volatile语义的当前值,并设置变量volatile语义的值为update。
    \item long getAndAddLong(Object o, long offset, long delta)
    
获取对象obj中偏移量为offset的变量volatile语义的当前值,并设置变量值为原始值+addValue。
\end{itemize}

\subsection{指令重排序}

