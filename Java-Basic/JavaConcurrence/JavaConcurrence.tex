\documentclass{PionpillNote-book}
\usetikzlibrary {intersections,through,arrows.meta,graphs,shapes.misc,positioning,shapes.misc,positioning,calc}
\usetikzlibrary{animations}
\usetikzlibrary {shapes.geometric}
\usetikzlibrary {animations}
\usetikzlibrary {shapes.multipart}
\usetikzlibrary {positioning}
\usetikzlibrary {fit,shapes.geometric}
\usetikzlibrary {automata}
\usetikzlibrary {quotes}
\usetikzlibrary {matrix}
\usetikzlibrary {backgrounds}
\usetikzlibrary {scopes}
\usetikzlibrary {calc}
\usetikzlibrary {intersections}
\usetikzlibrary {svg.path}
\usetikzlibrary {decorations}
\usetikzlibrary {patterns}
\usetikzlibrary {decorations.pathmorphing}
\usetikzlibrary {shadows}
\usetikzlibrary {bending}

\title{Java Concurrence}
\author{
    Pionpill \footnote{笔名:北岸,电子邮件:673486387@qq.com,Github:\url{https://github.com/Pionpill}} \\
    本文档为作者学习 Java 并发时的笔记。\\
}

\date{\today}

\begin{document}

\pagestyle{plain}
\maketitle

\noindent\textbf{前言:}

笔者为软件工程系在校本科生,有计算机学科理论基础(操作系统,数据结构,计算机网络,编译原理等),本人在撰写此笔记时已有 Java 开发经验,基础知识不再赘述。

本文需要的前置知识: 扎实的 Java 基础,对 JVM 具有一定了解。

本文各部分内容如下:
\begin{itemize}
    \item Java 并发编程: 并发编程基础,Java 实现并发的原理。参考 <<Java 并发编程之美>>\footnote{<<Java 并发编程之美>>: 翟陆续 2018 年出版} 。
\end{itemize}

本人的编写及开发环境如下:
\begin{itemize}
    \item Java: Java17
    \item OS: Windows11
\end{itemize}

\date{\today}
\newpage

\tableofcontents

\newpage

\setcounter{page}{1} 
\pagestyle{fancy}

\part{Java 并发编程}
\chapter{并发编程基础}
\import{Parts/Part-Basic/Thread}{Thread.tex}
\import{Parts/Part-Basic/Thread}{Lock.tex}
\end{document}