\section{并发}

\subsection{什么是线程}

进程是资源分配的基本单位,是程序执行的实例。程序运行时系统会创建进程并为它分配资源,然后把该进程放入进程就绪队列,进程调度器选中它的时候就会为它分配CPU时间,程序开始真正运行。

线程是一条执行路径,是程序执行时的最小单位,它是进程的一个执行流,是CPU调度和分派的基本单位,一个进程可以由很多个线程组成,线程间共享进程的所有资源,每个线程有自己的堆栈和局部变量。线程由CPU独立调度执行,在多CPU环境下就允许多个线程同时运行。同样多线程也可以实现并发操作,每个请求分配一个线程来处理。

一个正在运行的软件(如迅雷)就是一个进程,一个进程可以同时运行多个任务( 迅雷软件可以同时下载多个文件,每个下载任务就是一个线程), 可以简单的认为进程是线程的集合。

我们需要知道的是,Java 中使用线程可以并发运行,提高程序效率。可以通过如下方式启动一个线程:
\begin{Java}
Thread thread = new Thread(Runnable target) // 构造线程,并调用目标的 run() 方法。
thread.start()  // 启用线程,调用 run 方法。这个方法会立即返回。新线程并发运行。
\end{Java}

\subsection{线程状态}

线程有如下六种状态:
\begin{itemize}
    \item New: 新建
    \item Runnable: 可运行
    \item Blocked: 阻塞
    \item Waiting: 等待
    \item Timed waiting: 计时等待
    \item Terminated: 终止
\end{itemize}

要确定一个线程的当前状态,只需要调用 \texttt{getState} 方法。

\subsubsection{新建线程}

当用 \texttt{new} 操作符创建新线程时,线程还没有开始运行。他的状态是 \textit{New}。在线程运行之前还有一些基础工作要做。

\subsubsection{可运行线程}

一旦调用了 \texttt{start} 方法,线程就处于可运行状态。可运行的线程可能正在运行也可能不在运行。要由操作系统为线程提供具体的运行时间。(Java 没有将正在运行的线程作为一个单独的状态。仍然叫可运行状态。)

线程开始运行后,不一定始终保持运行,运行中的线程有时需要暂停,让其他线程有机会,这取决于操作系统的调用方式。抢占式操作系统给每一个可运行的线程一个时间片来执行任务。抢占式操作系统的具体操作请查阅相关资料。

虽然所有的桌面和服务器操作系统都使用抢占式调度。但是,像手机这样的小型设备可能使用协作式调度。在这样的设备中,一个线程只有在调用 yield 方法或者被阻塞或等待时才会失去控制权。

使用下面方法,可以让正在执行的线程向另一个线程交出运行权:
\begin{Java}
static void yield()
\end{Java}

\subsubsection{阻塞和等待线程}

当线程处于阻塞或等待状态时,它暂时是不活动的。不运行任何代码,而且消耗最少的资源。要由线程调度器激活这个线程。
\begin{itemize}
    \item 当一个线程试图获取一个内部的对象锁,而这个锁被其他线程占用,该线程会被阻塞。当所有线程都释放了这个锁,并且线程调度器允许该线程保持这个锁时,它将变成非阻塞状态。
    \item 当线程等待另一个线程通知调度器出现一个条件时,这个线程会进入等待状态。事实上阻塞和等待状态并没有太大的差别。
    \item 有几个方法有超时参数,调用这些方法会让线程进入计时等待状态。这一状态将一直保持到超时期满或者接收到适当的通知。
\end{itemize}

\subsubsection{终止线程}

线程会由于以下两个原因之一而终止:
\begin{itemize}
    \item \texttt{run} 方法正常退出,线程自然终止。
    \item 因为一个没有捕获的异常终止了 \texttt{run} 方法,使线程意外终止。
\end{itemize}

\subsection{线程属性}

\subsubsection{中断线程}

当线程的 \texttt{run} 方法执行方法体中的最后一条语句后再执行 \texttt{return} 语句返回时,或者出现了方法中没有捕获的异常时,线程将终止。除了已经废除的 \texttt{stop} 方法,没有办法可以强制线程终止。不过,\texttt{interrupt} 方法可以用来请求终止一个线程。

当对一个线程调用 \texttt{interrupt} 方法时,就会设置线程的\texttt{中断状态}。这是每个线程都有的 \texttt{boolean} 标志。每个线程都应该不时地检查这个标志,以判断线程是否被中断。

要想得出是否设置了中断状态,首先调用静态的 \texttt{Thread.currentThread} 方法获得当前线程,然后调用 \texttt{isInterrupted} 方法:

\begin{Java}
while (!Thread.currentThread().isInterrupted() && ...)
\end{Java}

但是,如果线程被阻塞,就无法检查中断状态。这里就要引入 \texttt{InterruptedException} 异常。

没有任何语言要求被中断的线程应当终止。中断一个线程只是要引起它的注意。被中断的线程可以决定如何相应中断。某些线程非常重要,所以应该处理这个异常,然后再继续执行。但更普遍的是,线程只需要将中断解释为一个终止请求。这种线程的 \texttt{run} 方法具有如下形式:

\begin{Java}
Runnable r = () -> {
    try {
        ...
        while (!Thread.currentThread().isInterrupted() && ...) {
            do work...
        }
    } catch (InterruptedException e) {
        // thread was interrupted during sleep or wait
    } finally {
        // cleanup if required
    }
};
\end{Java}

如果再每次工作迭代之后都调用 \texttt{sleep} 方法, \texttt{isInterrupted} 检查既没有必要也没有用处。如果设置了中断状态,此时倘若调用 \texttt{sleep} 方法,他不会休眠。实际上,它会清除中断状态并抛出 \texttt{InterruptedException}。因此,如果循环调用了 \texttt{sleep},不要检测中断状态,而应该捕获 \texttt{InterruptException} 异常。

\begin{Java}
Runnable r = () -> {
    try {
        ...
        while (...) {
            ...
            Thread.sleep(delay);
        }
    } catch (InterruptedException e) {
        // thread was interrupted during sleep
    } finally {
        // cleanup if required
    }
};
\end{Java}

\subsubsection{守护线程}

可以通过

\begin{Java}
t.setDaemon(true);
\end{Java}

将一个线程转换为守护线程,这样一个线程并没有什么魔力。守护线程的唯一用途是为其他线程提供服务。当只剩下守护线程时,虚拟机就会退出,因为如果只剩下守护线程,就没必要继续运行程序了。

\subsubsection{线程名}

默认情况下,线程名为: \texttt{Thread-2}。可以用 \texttt{setName} 方法为线程设置任何名字:

\begin{Java}
var t = new Thread(runnable);
t.setName("Web thread");
\end{Java}

\subsubsection{未捕获异常的处理器}

线程的 \texttt{run} 方法不能抛出任何检查型异常,但是,非检查型异常可能会导致线程终止。在这种情况下,线程会死亡。

不过,对于可以传播的异常,并没有任何 \texttt{catch} 子句。实际上,在线程死亡之前,异常会传递到一个用于处理未捕获异常的处理器。

这个处理器必须属于一个实现了 \texttt{Thread.UncaughtExceptionHandler} 接口的类。这个接口只有一个方法:

\begin{Java}
void uncaughtException(Thread t,Throwable e)
\end{Java}

可以用 \texttt{setUncaughtExceptionHandler} 方法为任何线程安装一个处理器。也可以用 \texttt{Thread} 类的静态方法 \texttt{setDefaultUncaughtExceptionHandler} 为所有线程安装一个默认的处理器。替代处理器可以使用日志 API 将未捕获异常的报告发送到一个日志文件。

\subsubsection{线程优先级}

一般情况下,一个线程会继承构造它的那个线程的优先级。可以用 \texttt{setPriority} 方法提高或降低任何一个线程的优先级。可以设置为:
\begin{itemize}
    \item \texttt{MIN\_PRIORITY}: 1
    \item \texttt{MAX\_PRIORITY}: 10
    \item \texttt{NORM\_PRIORITY}: 5
\end{itemize}

线程的优先级相当依赖于操作系统,Windows 有 7 个优先级,Java 会对优先级进行映射。有些 Linux 系统则只有一个优先级。

\subsection{同步}

在大多数实际的多线程应用中,两个或两个以上的线程需要共享对同一数据的存取。如果两个线程存取同一个对象,并且每个线程分别调用了一个修改该对象状态的方法,会导致对象被破坏,这种情况通常称为\textit{竞态条件}。

为了避免多线程破坏共享数据,必须学习如何同步存取。

\subsubsection{静态条件详解}

当两个线程同时对公共数据进行存取时,公共数据可能会被破坏。例如多个线程执行以下操作:

\begin{Java}
accounts[to] += amount;
\end{Java}

问题的关键在于原子操作,这个指令可能如下处理:
\begin{itemize}
    \item 将 \texttt{accounts[to]} 加载到寄存器。
    \item 增加 \texttt{amounts}。
    \item 将结果写回 \texttt{accounts[to]}。
\end{itemize}

假定第一个线程执行前两个步骤,然后它的运行权被抢占。再假设第2个线程被唤醒,更新 \texttt{accounts} 数组中的同一个元素。然后,第一个线程被幻想并完成其第3步。这个动作会抹去第2个线程所做的更新,就会出错。

\subsubsection{锁对象}

有两种机制可防止并发访问代码块。Java 语言提供了一个 \texttt{synchronized} 关键字来达到这一目的。Java5 引入了 \texttt{ReentrantLock} 类。

用 \texttt{ReentrantLock} 保护代码块的基本结构如下:
\begin{Java}
myLock.lock();
try {
    // critical section
} finally {
    myLock.unlock();
}
\end{Java}

这个结构确保任何时刻只有一个线程进入临界区。一旦一个线程锁定了锁对象,其他任何线程都无法通过 \texttt{lock} 语句。当其他线程调用 \texttt{lock} 时,它们会暂停,知道第一个线程释放这个锁对象。

\fbox{
    \parbox{0.87\textwidth}{
        \begin{warning}
            必须要将 \texttt{unlock} 操作包括在 \texttt{finally} 子句中,这一点至关重要。如果在临界区的代码抛出一个异常,锁必须释放。否则,其他线程将永远阻塞。\\ 使用锁时,就不能使用 \texttt{try-with-resources} 语句。他会在首部声明一个新变量。但是如果使用一个锁,可能想使用多个线程共享那个变量。
        \end{warning}
    }
}

理论上,给不同的线程加锁,每个线程会维护不同的锁对象,两个线程都不会阻塞。本应该如此,因为线程在操作不同的实例时,线程之间不会相互影响。

我们使用的锁称为重入(reentrant)锁,因为线程可以反复获得已拥有的锁。锁有一个持有计数来跟踪对 \texttt{lock} 方法的嵌套调用。线程每一次调用 \texttt{lock} 后都要调用 \texttt{unlock} 来释放锁。由于这个特性,被一个锁保护的代码可以调用另一个使用相同锁的方法。

\subsubsection{条件对象}

通常,线程进入临界区后却发现只有满足了某个条件后再能执行。可以使用一个\textit{条件对象}来管理那些已经进入了一个锁却不能做有用工作的线程。

现在我们考虑这样一种情况:
\begin{Java}
if (bank.getBalance(from) >= amount)
    bank.transfer(from, to, amount);
\end{Java}

如果我们在判断语句之后,被其他线程占用,随后的操作会产生错误的结果,因此,这里需要加锁:
\begin{Java}
public void transfer(int from, int to, int amount) {
    bankLock.lock();
    try {
        while(accounts[from] < amount) {
            //wait
        }
        // transfer funds
    } finally {
        bankLock.unlock();
    }
}
\end{Java}

如果账户中没有足够的资金,线程会等待其他线程向账户中添加资金。现在这个线程获得了 \texttt{bankLock} 的排他性访问,因此别的线程没有几乎进行,会一直等待下去。

一个锁对象可以有一个或多个相关联的条件对象。可以用 \texttt{newCondition} 方法获得一个条件对象。

\begin{Java}
class Bank {
    private Condition sufficientFunds;
    ...
    public Bank() {
        ...
        sufficientFunds = bankLock.newCondition();
    }
}
\end{Java}

如果 \texttt{transfer} 方法发现资金不足,它会调用:

\begin{Java}
sufficientFunds.await();
\end{Java}

当前线程现在暂停,并放弃锁。这就允许另一个线程执行。等待获得锁的线程和已经调用了 \texttt{await} 方法的线程存在本质上的不同。一旦一个线程调用了 \texttt{await} 方法,它就进入了这个条件的 \textit{等待集}。当锁可用时,该线程并不会变为可运行状态。实际上,它人人保持非活动状态,直到另一个线程在同一条件上调用 \texttt{signalAll} 方法。

这个调用会重新激活等待这个条件的所有线程。当这些线程从等待集中移出时,它们再次成为可运行的线程,调度器最终将再次将它们激活。同时,它们会尝试重新进入该对象。一旦锁可用,它们中的某个线程将从 \texttt{await} 调用返回,得到这个锁,并从之前暂停的地方继续执行。

此时,线程应该再次测试条件。不能保证现在一定满足条件,\texttt{signalAll} 方法仅仅是通知等待的线程: 现在有可能满足条件,值得再次检查条件。

最终需要有某个其他线程调用 \texttt{signalAll} 方法,这一点至关重要。当一个线程调用 \texttt{await} 时,他没有办法重新自行激活。它寄希望于其他线程。如果没有其他线程来重新激活等待的线程,它就永远不再运行了。这将导致 \textit{死锁} 现象。

\subsubsection{\texttt{synchronized} 关键字}

\texttt{Lock} 和 \texttt{Condition} 接口允许程序员充分控制锁定。但大多数情况下可以使用 Java 语言内置的一种机制。从 1.0 版本开始,Java 中每个对象都有一个内部锁。若果一个方法声明有 \texttt{synchronized} 关键字,那么对象的锁将保护整个方法。也就是说,要调用这个方法,线程必须获得内部对象锁。

下面代码等价:
\begin{Java}
public synchronized void method () {
    // method body
}

public void method() {
    this.intrinsicLock.lock();
    try {
        // method body
    } finally {
        this.intrinsicLock.unlock();
    }
}
\end{Java}

内部对象锁只有一个关联条件。\texttt{wait} 方法将一个线程增加到等待集中,\texttt{notifyAll/notify} 方法可以接触等待线程的阻塞。换句话说,调用 \texttt{wait} 或 \texttt{notifyAll} 等价于:

\begin{Java}
intrinsicCondition.await();
intrinsicCondition.signalAll();
\end{Java}

\fbox{
    \parbox{0.87\textwidth}{
        \begin{notice}
            \texttt{wait, notifyAll, notify} 是 \texttt{Object} 类的 \texttt{final} 方法。\texttt{Condition} 方法必须命名为 \texttt{await, signalAll, signal},从而不会与那些方法发生冲突。
        \end{notice}
    }
}

关于线程锁的做法:
\begin{itemize}
    \item 优先使用 \texttt{java.util.concurrent} 包中的某种机制,它会处理所有的锁。
    \item 其次使用 \texttt{synchronized} 关键字。
    \item 最后考虑 \texttt{Lock/Condition} 结构提供的额外能力。
\end{itemize}

\subsubsection{同步块}

每一个 Java 对象都有一个锁。线程可以通过调用同步方法获得锁。还有另一种机制可以获得锁: 进入一个同步块。

\begin{Java}
synchronized (obj) {
    // critical section
}
\end{Java}

它会获得 \texttt{obj} 的锁,书上讲的比较抽象,建议查看这篇文章: \url{https://blog.csdn.net/csdnnews/article/details/82321777}。

\subsubsection{线程局部变量}

线程在共享变量时存在一定的风险。有时可能要避免共享变量,使用 \texttt{ThreadLocal} 辅助类可以为各个线程提供各自的实例。

\subsection{任务和线程池}

构造一个新的线程开销有些大,因为这涉及到与操作系统的交互。如果程序中创建了大量生命期很短的线程,那么不应该吧每个任务映射到一个单独的线程,而应该使用线程池。线程池中包含许多准备运行的线程。为线程池提供一个 \texttt{Runnable} 就会有一个线程调用 \texttt{run} 方法。当 \texttt{run} 方法退出时,这个线程不会死亡,而是留在池中准备为下一个请求提供服务。

\subsubsection{\texttt{Callable} 与 \texttt{Future}}

\texttt{Runnable} 封装一个异步运行的任务,可以把它想象成一个没有参数和返回值的异步方法。\texttt{Callable} 与 \texttt{Runnable} 类似,但是有返回值。\texttt{Callable} 接口是一个参数化的类型,只有一个方法 \texttt{call}。

\begin{Java}
public interface Callable<V> {
    V call() throws Exception;
}
\end{Java}

类型参数是返回值的类型。例如,\texttt{Callable<Integer>} 表示一个最终返回 \texttt{Integer} 对象的异步计算。

\texttt{Future} 保存异步计算的结果。可以启动一个计算,将 \texttt{Future} 对象交给某个线程,然后忘掉它。这个 \texttt{Future} 对象的所有者在结果计算好之后就可以获得结果。

