\section{集合}
\subsection{Java 集合框架}
\subsubsection{集合接口与实现分离}

\fbox{
    \parbox{0.87\textwidth}{
        \begin{warning}
            这一节讲了很多数据结构与算法,这不是 Java 原有的特性,只是用 Java 语言进行了实现,故省去了很多。
        \end{warning}
    }
}

与现代的数据结构类库的常见做法一样,Java 集合类库也将接口与是西安分离。例如队列接口:

\begin{Java}
public interface Queue<E> {
    void add(E element);
    E remove();
    int Size();
}
\end{Java}

我们可以使用循环数组或者链表实现队列。但使用者并不关心队列的具体实现方法。

\subsubsection{\texttt{Collection} 接口}

在 Java 类库中,集合类的基本接口是 \texttt{Collection} 接口。

\begin{Java}
public interface Collection<E> {
    boolean add(E element);
    Iterator<E> iterator();
    ...
}
\end{Java}

其中,\texttt{add} 方法用于向集合中添加元素。如果添加元素确实改变了集合就返回 \texttt{true}; 如果集合没有发生变化就返回 \texttt{false}。

\texttt{iterator} 方法用于返回一个实现了 \texttt{Iterator} 接口的对象。可以使用这个迭代器对象依次访问集合中的元素。

\subsubsection{迭代器}

\texttt{Iterator} 接口包含 4 个方法:

\begin{Java}
public interface Iterator<E> {
    E next();
    boolean hasNext();
    void remove();
    default void forEachRemaining(C)
}
\end{Java}

通过反复调用 \texttt{next} 方法,可以逐个访问集合中的每个元素。但是,如果到达了集合的末尾, \texttt{next} 方法将抛出一个 \texttt{NoSuchElementException}。因此在调用之前要用 \texttt{hasNext} 进行判断。如果迭代器对象还有多个可以访问的元素,这个方法就返回 \texttt{true}。如果想要查看集合中的所有元素,就请求一个迭代器,当 \texttt{hasNext} 返回 true 时就反复地调用 \texttt{next} 方法。

编译器简单地将 ``for each'' 循环转换为带有迭代器的循环,可以处理任何实现了 \texttt{Iterable} 接口的对象,这个接口只包含一个抽象方法:

\begin{Java}
public interface Iterable<E> {
    Iterator<E> iterator();
    ...
}
\end{Java}

\subsubsection{集合框架中的接口}

集合有两个基本接口: \texttt{Collection} 和 \texttt{Map}。我们可以直接插入元素:

\begin{Java}
boolean add(E element)
\end{Java}

由于映射包含键值对,更常见的,使用如下方法:
\begin{Java}
V put(K key, V value)
V get(K key)
\end{Java}

\subsection{具体集合}

\subsubsection{链表与数组列表}

链表结构比较复杂,Java 中的链表都是双向链表。我们不能对链表(\texttt{LinkedList})进行一边读取一边写入的操作,这往往会导致严重的错误。我们需要遵守一个规则: 可以根据需要为一个集合关联多个迭代器,前提是这些迭代器只能读取集合。或者,可以关联一个能同时读写的迭代器。

数组列表相对简单,但是如果对数组列表进行增减操作,会消耗大量的时间。

还有一种快速查询元素的集合叫散列表,我已经在别的笔记写过很多次,这里不写了。此外还有树结构,队列。都是数据结构的内容,这里省略。

\subsection{映射}

映射用来存放键/值对。如果提供了键就能查找到值。例如,可以存储一个员工记录表,其中键为员工 ID,值为 \texttt{Employee} 对象。

\subsubsection{基本映射操作}

Java 类库为映射提供了两个通用的实现: \texttt{HashMap} 和 \texttt{TreeMap}。这两个类都实现了 \texttt{Map} 接口。 散列映射与树映射的差别类似于散列表与树的差别。

映射的基础内容没什么好讲的,操作类似于 Python 中的字典。

\subsubsection{弱散列映射}

设计 \texttt{WeakHashMap} 类是为了解决这样一个有趣的问题。如果有一个值,它对应的键已经不再在程序中的任何地方使用。将会出现什么情况?假定对某个键的最后一个引用已经消失,那么不再有任何途径可以引用这个值的对象。但是,由于程序中的任何部分不会再有这个键,所以无法从映射中删除这个键值对。

为什么垃圾回收不能删除它呢? 垃圾回收器会跟踪活动的对象。只要映射对象是活动的,其中的所有桶也是活动的,它们就不能被回收。因此,需要由程序负责从长期存活的映射表中删除那些无用的值。或者使用 \texttt{WeakHashMap},当对键的唯一引用来自散列表映射条目时,这个数据结构将被回收。

\subsubsection{链接散列集与映射}

\texttt{LinkedHashSet} 和 \texttt{LinkedHashMap} 类会记住插入元素项的顺序。这样就可以避免散列表中的项看起来是随机的。在表中插入元素时,就会并入到双向链表中。

\subsubsection{标识散列映射}

类 \texttt{IdentityHashMap} 有特殊的用途。在这个类中,键的散列值不是用 \texttt{hashCode} 函数计算的,而是用 \texttt{System.identityHashCode} 方法计算的。这是 \texttt{Object.hashCode} 根据对象的内存地址计算散列码时所用的方法。而且,在对两个对象进行比较时,\texttt{IdentityHashMap} 类使用 \texttt{==},而不使用 \texttt{equals}。

也就是说,不同的键对象即使内容相同,也被视为不同的对象。 

\newpage