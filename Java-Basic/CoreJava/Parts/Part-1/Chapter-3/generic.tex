\section{泛型程序设计}

\subsection{类型参数的好处}

在 Java 中增加泛型类之前,泛型程序设计都是通过继承实现的。 \texttt{ArrayList} 类只维护一个 \texttt{Object} 引用的数组:

\begin{Java}
public class ArrayList {
    private Object[] elementData;
    ...
    public Object get(int i) {...}
    public void add(Object o) {...}
}
\end{Java}

这种方法存在两个问题: 当取值时必须进行强制类型转换;没有错误检查,可以将任何类型的值加入到数组中。

泛型提供了一个更好的解决方案: \textit{类型参数}(type parameter)。\texttt{ArrayList} 类有一个类型参数用来指示元素的类型:

\begin{Java}
var files = new ArrayList<String>();
\end{Java}

这样的好处如下:
\begin{itemize}
    \item 代码具有更好的可读性。
    \item 编译器会进行类型检查。
\end{itemize}

使用泛型编程有多容易,编写泛型类就有多困难。因为编写者需要预判各种各样的类作为类型参数。

\subsection{定义简单泛型}

\subsubsection{泛型类}

泛型类就是有一个或多个类型变量的类:
\begin{Java}
public class Pair<T> {
    private T first;
    private T second;

    public Pair() {first = null; second = null;}
    public Pair(T first, T second) {this.first = first; this.second = second;}

    public T getFirst() {return first;}
    public T getSecond() {return second;}

    public void setFirst(T newValue) {first = newValue;}
    public void setSecond(T newValue) {second = newValue;}
}
\end{Java}

类似的,可以定义多个字段类型:

\begin{Java}
public class Pair<T,U> {...}
\end{Java}

\fbox{
    \parbox{0.87\textwidth}{
        \begin{notice}
            Java 库使用变量 E 表示集合的元素类型,K/V 分别便是键和值。T 表示任意类型。
        \end{notice}
    }
}

\subsubsection{泛型方法}

在泛型方法中,需要把参数类型放在修饰符后面,返回值类型前面。定义一个带参数类型的方法:

\begin{Java}
public static <T> T getMiddle(T... a) {...}
\end{Java}

在调用泛型方法时,可以明确参数类型,也可以不写参数类型,因为编译器会自动推导出参数类型。

\begin{Java}
String middle = ArrayAlg.<String>getMiddle("John","Q.");
String middle = ArrayAlg.getMiddle("John","Q.");
\end{Java}

\subsubsection{类型变量的限定}

有时,类或方法需要对类型变量加以约束,比如我们希望类型 T 实现了 \texttt{compareTo} 方法,可以这样写:

\begin{Java}
public static <T extends Comparable> T min(T[] a) ...
\end{Java}

如果需要多个限制条件,可以这样写:

\begin{Java}
T extends Comparable & Serializable
\end{Java}

\subsection{泛型代码与虚拟机}

虚拟机没有泛型类型对象,所有对象都属于普通类。

\subsubsection{类型擦除}

无论如何定义一个泛型类型,都会自动提供一个相应的原始类型。这个原始类型的名字就是去掉类型参数后的泛型类型名。类型变量会被擦除,并替换为限定类型(或者,无限定变量被替换为 Object)。

例如:
\begin{Java}
public class Pair {
    private Object first;
    private Object second;

    public Pair() {first = null; second = null;}
    public Pair(Object first, Object second) {this.first = first; this.second = second;}

    public Object getFirst() {return first;}
    public Object getSecond() {return second;}

    public void setFirst(Object newValue) {first = newValue;}
    public void setSecond(Object newValue) {second = newValue;}
}
\end{Java}

因为 T 没有被限定,所以被替换为了 \texttt{Object}。如果有限定,比如 \texttt{Comparable}。那么类型将不是 \texttt{Object} 而是 \texttt{Comparable}。

\subsubsection{转换泛型表达式}

编写一个泛型方法调用时,如果擦除了返回类型,编译器会插入强制类型转换。例如:
\begin{Java}
Pair<Employee> buddies = ...;
Employee buddy = buddies.getFirst();
\end{Java}

\texttt{getFirst} 擦除类型后的返回类型是 \texttt{Object}。编译器自动插入转换到 \texttt{Employee} 的强制类型转换。

\subsubsection{转换类型方法}

在类型擦除的过程中,会遇到很多复杂的问题。比如,我在代码中使用了 \texttt{super} 调用父类的方法,而类型擦除后,我们无法知道这个类型的父类是谁,也即无法保持多态。

Java 虚拟机会使用名为桥方法的方式重写或者引用等方式保持多态。总而言之,虚拟机帮你避免类型擦除与多态发生冲突。

对于 Java 泛型,我们需要记住以下几个事实:
\begin{itemize}
    \item 虚拟机中没有泛型,只有普通的类和方法。
    \item 所有类型参数都会被替代。
    \item 会合成桥方法保持多态。
    \item 为了保持安全,必要时插入强制类型转换。
\end{itemize}

\subsection{限制与局限性}

\subsubsection*{不能用基本类型实例化类型参数}

基本类型不能作为类型参数,应该 使用对应的包装器,如 \texttt{double -> Double}。这是因为 \texttt{Object} 无法存储 \texttt{double} 值。

\subsubsection*{运行时类型查询只适用于原始类型}

虚拟机中的对象总有一个特定的非泛型类型。因此,所有类型查询只产生原始类型:

\begin{Java}
if (a instanceof Pair<String>) //Error
if (stringPair.getClass() == employeePair.getClass()) // equal
\end{Java}

\subsubsection*{不允许创建参数化类型的数组}

例如:
\begin{Java}
var table = new Pair<String>[10];
\end{Java}

擦除之后,\texttt{table} 类型是 \texttt{Pair[]},转换为 \texttt{Object[]}。

数组会记住它的元素类型,如果试图存储其他类型的元素,会抛出错误。

尽管可以通过数组存储的检查,但仍会导致一个类型错误。因此不允许创建参数化类型的数组。

\subsubsection*{Varargs 警告}

varargs 拆开来: var args。即多参数警告。具体为向参数个数可变的方法传递泛型类型的实例。

在传递参数个数可变的实例时,实际上传的就是参数数组。这就违背了之前的规则,但这种情况下,我们只会得到一个警告,而不是错误。

可以使用添加注解 \texttt{@SupportWarnings("unchecked")} 的方式消除警告。Java7 中可以使用 \texttt{@SafeVarargs}。

\subsubsection*{不能实例化类型变量}

考虑这样一个代码:
\begin{Java}
public Pair() {first = new T(); second = new T();} // Error
\end{Java}

由于 \texttt{T()} 会被转换为 \texttt{Object()} 所以上述代码肯定会出问题。

同样的也不能实例化泛型数组。

\subsubsection*{其他缺陷}

还有一些不是很常见的缺陷:

\begin{itemize}
    \item 不能在静态字段或方法中引用类型变量。
    \item 不能抛出或捕获泛型类的实例。
    \item 可以取消对检查型异常的检查。
    \item 注意擦除后的冲突。
\end{itemize}

\subsection{泛型类型的继承规则}

只需要记住,无论 S 与 T 有什么继承关系,Pair<S> 与 Pair<T> 都没有任何关系。也即泛型类型之间没有任何关系。

\subsection{通配符类型}

\subsubsection{通配符上下界}

在通配符类型中,允许类型参数发生变化。

\begin{Java}
Pair <? extends Employee>
\end{Java}

表示任何泛型 \texttt{Pair} 类型,它的类型参数是 \texttt{Employee} 的子类,如 \texttt{Pair<Manager>}。

其中 ? 代表类型不限制,\texttt{extends} 关键字则制定了类型的上界。类似的,还有 \texttt{super} 关键字,限制为所有的超类型。

\subsubsection{无限定通配符}

无限定通配符就是 <?> 代表不确定类型\footnote{参考文献:\url{https://blog.csdn.net/yu_duan_hun/article/details/123975211}}。

\newpage